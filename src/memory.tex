\newenvironment{tentative}
{ \vspace{-0.2in}
  \begin{quotation}
  \noindent
  \color{red}MEMORY MODEL TASK GROUP TO-DO

  \small \em
  \rule{\linewidth}{1pt}\\
}
{ 
  \end{quotation}
  \vspace{-0.2in}
}
\lstdefinelanguage{alloy}{
  morekeywords={abstract, sig, extends, pred, fun, fact, no, set, one, lone, let, not, all, iden, some, run, for},
  morecomment=[l]{//},
  morecomment=[s]{/*}{*/},
  commentstyle=\color{green!40!black},
  keywordstyle=\color{blue!40!black},
  moredelim=**[is][\color{red}]{@}{@},
  escapeinside={!}{!},
}
\lstset{language=alloy}
\lstset{aboveskip=0pt}
\lstset{belowskip=0pt}

\newcommand{\diagram}{(picture coming soon)}

\chapter{RVWMO Explanatory Material}
\label{sec:explanation}
This section provides more explanation for the RVWMO memory model, using more informal language and concrete examples.
These are intended to clarify the meaning and intent of the axioms and preserved program order rules.
%In case of any discrepancy between the informal descriptions here and the formal descriptions elsewhere, the formal definitions should be considered authoritative.

\section{Why RVWMO?}
\label{sec:whynottso}

Memory consistency models fall along a loose spectrum from weak to strong.
Weak memory models allow more hardware implementation flexibility and deliver arguably better performance, performance per watt, power, scalability, and hardware verification overheads than strong models, at the expense of a more complex programming model.
%Models which are too weak may not even be properly analyzable with modern formal analysis techniques.
Strong models provide simpler programming models, but at the cost of imposing more restrictions on the kinds of (non-speculative) hardware optimizations that can be performed in the pipeline and in the memory system, and in turn imposing some cost in terms of power, area overhead, and verification burden.

RISC-V has chosen the RVWMO memory model, which is a variant of release consistency.
This places it in between the two extremes of the memory model spectrum.
%It is not as weak as the Power memory model, and this buys back some programming model simplicity without giving up very much in terms of performance.
%RVWMO is also not as restrictive as RVTSO, and hence it remains weak enough to ensure that implementations can be performant and scalable without incurring huge hardware complexity overheads.
%RVWMO is similar to the ARMv8 memory model in this regard.
The RVWMO memory model enables architects to build simple implementations, aggressive implementations, implementations embedded deeply inside a much larger system and subject to complex memory system interactions, or any number of other possibilities, all while simultaneously being strong enough to support programming language memory models at high performance.

%The risk of a weak memory model lies in the complexity of the programming model.
%Buggy code which ``just worked'' on stronger implementations may well break on more aggressive implementations due to the bugs simply not manifesting on the stronger-than-necessary implementations.
%For these situations, though, the root cause is the bug in the original software, not the memory model itself.
%The risk of finding short-term bugs in code ported from other architectures is outweighed by the long-term benefits that the weak memory model delivers more generally.

To facilitate the porting of code from other architectures, some hardware implementations may choose to implement the Ztso extension, which provides stricter RVTSO ordering semantics by default.
Code written for RVWMO is automatically and inherently compatible with RVTSO, but code written assuming RVTSO is not guaranteed to run correctly on RVWMO implementations.
In fact, most RVWMO implementations will (and should) simply refuse to run RVTSO-only binaries.
Each implementation must therefore choose whether to prioritize compatibility with RVTSO code (e.g., to facilitate porting from x86) or whether to instead prioritize compatibility with other RISC-V cores implementing RVWMO.

Some fences and/or memory ordering annotations in code written for RVWMO may become redundant under RVTSO; the cost that the default of RVWMO imposes on Ztso implementations is the incremental overhead of fetching those fences (e.g., {\tt fence~r,rw} and {\tt fence rw,w}) which become no-ops on that implementation.
However, these fences must remain present in the code to ensure compatibility with non-Ztso implementations.

%Most software is also fully compatible with weak memory models.
%C/C++, Java, and Linux, to name some of the most notable and more formally analyzed examples, are all entirely compatible with weak non-atomic memory models, as all are designed to run not just on x86 but also on ARM, Power, and many other architectures.
%It is true that some code, e.g., code ported from x86, does sometimes (correctly or incorrectly) assume a stronger model such as TSO.
%For such code, the RVWMO memory model provides a means for restoring TSO to sections of code through fences and atomics with {\tt .aq} and {\tt .rl} bits in the ``A'' extension, until such code can be ported to RVWMO over time.

%dDesigners who wish to provide drop-in compatibility with x86 code can also implement the Ztso extension which enforces RVTSO.

\section{Litmus Tests}
The explanations in this chapter make use of {\em litmus tests}, or small programs designed to test or highlight one particular aspect of a memory model.
Figure~\ref{fig:litmus:sample} shows an example of a litmus test with two harts.
For this figure (and for all figures that follow in this chapter), we assume that {\tt s0}--{\tt s2} are pre-set to the same value in all harts.
As a convention, we will assume that {\tt s0} holds the address labeled {\tt x}, {\tt s1} holds {\tt y}, and {\tt s2} holds {\tt z}, where {\tt x}, {\tt y}, and {\tt z} are different memory addresses.
This figure shows the same program twice: on the left in RISC-V assembly, and again on the right in graphical form.

\begin{figure}[h!]
  \centering
  {
    \tt\small
    \begin{tabular}{cl||cl}
    \multicolumn{2}{c}{Hart 0} & \multicolumn{2}{c}{Hart 1} \\
    \hline
          & $\vdots$    &     & $\vdots$    \\
          & li t1,1     &     & li t4,4     \\
      (a) & sw t1,0(s0) & (e) & sw t4,0(s0) \\
          & $\vdots$    &     & $\vdots$    \\
          & li t2,2     &     &             \\
      (b) & sw t2,0(s0) &     &             \\
          & $\vdots$    &     & $\vdots$    \\
      (c) & lw a0,0(s0) &     &             \\
          & $\vdots$    &     & $\vdots$    \\
          & li t3,3     &     & li t5,5     \\
      (d) & sw t3,0(s0) & (f) & sw t5,0(s0) \\
          & $\vdots$    &     & $\vdots$    \\
    \end{tabular}
  }
  ~~~~
  \diagram
  \caption{A sample litmus test}
  \label{fig:litmus:sample}
\end{figure}

Litmus tests are used to understand the implications of the memory model in specific concrete situations.
For example, in the litmus test of Figure~\ref{fig:litmus:sample}, the final value of {\tt a0} in the first hart can be either 2, 4, or 5, depending on the dynamic interleaving of the instruction stream from each hart at runtime.
However, in this example, the final value of {\tt a0} in Hart 0 will never be 1 or 3; intuitively, the value 1 will no longer be visible at the time the load executes, and the value 3 will not yet be visible by the time the load executes.

We analyze this test and many others below.

\section{Explaining the RVWMO Rules}
In this section, we provide explanation and examples for all of the RVWMO rules and axioms.

\subsection{Preserved Program Order and Global Memory Order}
Preserved program order represents the set of intra-hart orderings that the hart's pipeline must ensure are maintained as the instructions execute, even in the presence of hardware optimizations that might otherwise reorder those operations.
Events from the same hart which are not ordered by preserved program order, on the other hand, may appear reordered from the perspective of other harts and/or observers.

Informally, the global memory order represents the order in which loads and stores perform.
The formal memory model literature has moved away from specifications built around the concept of performing, but the idea is still useful for building up informal intuition.
A load is said to have performed when its return value is determined.
A store is said to have performed not when it has executed inside the pipeline, but rather only when its value has been propagated to globally visible memory.
In this sense, the global memory order also represents the contribution of the coherence protocol and/or the rest of the memory system to interleave the (possibly reordered) memory accesses being issued by each hart into a single total order agreed upon by all harts.

The order in which loads perform does not always directly correspond to the relative age of the values those two loads return.
In particular, a load $b$ may perform before another load $a$ to the same address (i.e., $b$ may execute before $a$, and $b$ may appear before $a$ in the global memory order), but $a$ may nevertheless return an older value than $b$.
This discrepancy captures (among other things) the reordering effects of buffering placed between the core and memory.
For example, a younger load may read from a value in the store buffer, while an older load which appears before that store in program order may ignore that younger store and read an older value from memory instead.
To account for this, at the time each load performs, the value it returns is determined by the load value axiom, not just strictly by determining the most recent store to the same address in the global memory order, as described below.

\subsection{Store Buffering (Load Value Axiom)}
\begin{tabular}{p{1cm}|p{12cm}} &
Load value axiom:\loadvalueaxiom
\end{tabular}

Preserved program order is {\em not} required to respect the ordering of a store followed by a load to an overlapping address.
This complexity arises due to the ubiquity of store buffers in nearly all implementations.
Informally, the load may perform (return a value) by forwarding from the store while the store is still in the store buffer, and hence before the store itself performs (writes back to globally visible memory).
Any other hart will therefore observe the load as performing before the store.

\begin{figure}[h!]
  \centering
  {
    \tt\small
    \begin{tabular}{cl||cl}
    \multicolumn{2}{c}{Hart 0} & \multicolumn{2}{c}{Hart 1} \\
    \hline
          & li t1, 1    &     & li t1, 1    \\
      (a) & sw t1,0(s0) & (e) & sw t1,0(s1) \\
      (b) & lw a0,0(s0) & (f) & lw a2,0(s1) \\
      (c) & fence r,r   & (g) & fence r,r   \\
      (d) & lw a1,0(s1) & (h) & lw a3,0(s0) \\
    \end{tabular}
  }
  ~~~~
  \diagram
  \caption{A store buffer forwarding litmus test}
  \label{fig:litmus:storebuffer}
\end{figure}

Consider the litmus test of Figure~\ref{fig:litmus:storebuffer}.
When running this program on an implementation with store buffers, it is possible to arrive at the final outcome
{\tt a0=1, a1=0, a2=1, a3=0}
as follows:
\begin{itemize}
  \item (a) executes and enters the first hart's private store buffer
  \item (b) executes and forwards its return value 1 from (a) in the store buffer
  \item (c) executes since all previous loads (i.e., (b)) have completed
  \item (d) executes and reads the value 0 from memory
  \item (e) executes and enters the second hart's private store buffer
  \item (f) executes and forwards its return value 1 from (d) in the store buffer
  \item (g) executes since all previous loads (i.e., (f)) have completed
  \item (h) executes and reads the value 0 from memory
  \item (a) drains from the first hart's store buffer to memory
  \item (e) drains from the second hart's store buffer to memory
\end{itemize}
Therefore, the memory model must be able to account for this behavior.

To put it another way, suppose the definition of preserved program order did include the following hypothetical rule:
memory access $a$ precedes memory access $b$ in preserved program order (and hence also in the global memory order) if $a$ precedes $b$ in program order and $a$ and $b$ are accesses to the same memory location, $a$ is a write, and $b$ is a read.  Call this ``Rule X''.  Then we get the following:

\begin{itemize}
  \item (a) precedes (b): by rule X
  \item (b) precedes (d): by rule \ref{ppo:fence}
  \item (d) precedes (e): by the load value axiom.  Otherwise, if (e) preceded (d), then (d) would be required to return the value 1.  (This is a perfectly legal execution; it's just not the one in question)
  \item (e) precedes (f): by rule X
  \item (f) precedes (h): by rule \ref{ppo:fence}
  \item (h) precedes (a): by the load value axiom, as above.
\end{itemize}
The global memory order must be a total order and cannot be cyclic, because a cycle would imply that every event in the cycle happens before itself, which is impossible.
Therefore, the execution proposed above would be forbidden, and hence the addition of rule X would break the memory model.

Nevertheless, even if (b) precedes (a) and/or (f) precedes (e) in the global memory order, the only sensible possibility in this example is for (b) to return the value written by (a), and likewise for (f) and (e).  This combination of circumstances is what leads to the second option in the definition of the load value axiom.
Even though (b) precedes (a) in the global memory order, (a) will still be visible to (b) by virtue of sitting in the store buffer at the time (b) executes.
Therefore, even if (b) precedes (a) in the global memory order, (b) should return the value written by (a) because (a) precedes (b) in program order.
Likewise for (e) and (f).

\subsection{Same-Address Orderings, Part 1 (Rule~\ref{ppo:->st})}
\begin{tabular}{p{1cm}|p{12cm}} &
Rule \ref{ppo:->st}: \ppost
\end{tabular}

Same-address orderings where the latter is a store are straightforward: a load or store can never be reordered with a later store to an overlapping memory location.  From a microarchitecture perspective, generally speaking, it is difficult or impossible to undo a speculatively reordered store if the speculation turns out to be invalid, so such behavior is simply disallowed by the model.

Same-address load-load orderings are far more subtle; see Section~\ref{sec:ppo:rdw}.

\begin{comment}
The formal model captures this as follows:
\begin{itemize}
  \item (a) precedes (b) in preserved program order because both are stores to the same address, and (b) is a store (Rule~\ref{ppo:->st}).  Therefore, (c) cannot return the value written by (a), because (b) is a later store to the same address in both program order and the global memory order, and so returning the value written by (a) would violate the load value axiom.
  \item (c) precedes (d) in preserved program order because both are accesses to the same address, and (d) is a store.  (c) also precedes (d) in program order.  Therefore, (c) is not able to return the value written by (d), because neither option in the load value axiom applies.
\end{itemize}
\end{comment}

\subsection{Fences (Rule~\ref{ppo:fence})}\label{sec:mm:fence}
\begin{tabular}{p{1cm}|p{12cm}} &
Rule \ref{ppo:fence}: \ppofence
\end{tabular}

By default, the {\tt fence} instruction ensures that all memory accesses from instructions preceding the fence in program order (the ``predecessor set'') appear earlier in the global memory order than memory accesses from instructions appearing after the fence in program order (the ``successor set'').
However, fences can optionally further restrict the predecessor set and/or the successor set to  a smaller set of memory accesses in order to provide some speedup.
Specifically, fences have {\tt .pr}, {\tt .pw}, {\tt .sr}, and {\tt .sw} bits which restrict the predecessor and/or successor sets.
The predecessor set includes loads (resp.\@ stores) if and only if {\tt .pr} (resp.\@ {\tt .pw}) is set.
Similarly, the successor set includes loads (resp.\@ stores) if and only if {\tt .sr} (resp.\@ {\tt .sw}) is set.

The full RISC-V opcode encoding currently has nine non-trivial combinations of the four bits {\tt pr}, {\tt pw}, {\tt sr}, and {\tt sw}, plus one extra encoding ``{\tt fence.tso}'' which is expected to be added to facilitate mapping of ``acquire+release'' or RVTSO semantics.
The remaining seven combinations have empty predecessor and/or successor sets and hence are no-ops.
Of the ten non-trivial options, only six are commonly used in practice:
{\tt
\begin{itemize}
  \item fence rw,rw
  \item fence.tso \textrm{(i.e., a combined {\tt fence r,rw} $+$ {\tt fence rw,w})}
  \item fence rw,w
  \item fence r,rw
  \item fence r,r
  \item fence w,w
\end{itemize}
}
{\tt fence} instructions using any other combination of {\tt .pr}, {\tt .pw}, {\tt .sr}, and {\tt .sw} are reserved.  We strongly recommend that programmers stick to these six, as these are the best understood.

Finally, we note that since RISC-V uses a multi-copy atomic memory model, programmers can reason about fences and the {\tt .aq} and {\tt .rl} bits in a thread-local manner.  There is no complex notion of ``fence cumulativity'' as found in memory models which are not multi-copy atomic.

\subsection{Acquire/Release Ordering (Rules~\ref{ppo:acquire}--\ref{ppo:strongacqrel})}\label{sec:acqrel}
\begin{tabular}{p{1cm}|p{12cm}}
  & Rule \ref{ppo:acquire}: \ppoacquire \\
  & Rule \ref{ppo:release}: \pporelease \\
  & Rule \ref{ppo:strongacqrel}: \ppostrongacqrel \\
\end{tabular}

An {\em acquire} operation is used at the start of a critical section.  The general requirement for acquire semantics is that all loads and stores inside the critical section are up to date with respect to the synchronization variable being used to protect it.  In other words, an acquire operation requires load-to-load/store ordering.
Acquire ordering can be enforced in one of two ways: setting {\tt .aq}, which enforces ordering with respect to just the synchronization variable itself, or with a {\tt FENCE r,rw}, which enforces ordering with respect to all previous loads.  

\begin{figure}[h!]
  \centering\small
  \begin{verbatim}
          sd           x1, (a1)     # Random unrelated store
          ld           x2, (a2)     # Random unrelated load
          li           t0, 1        # Initialize swap value.
      again:
          amoswap.w.aq t0, t0, (a0) # Attempt to acquire lock.
          bnez         t0, again    # Retry if held.
          # ...
          # Critical section.
          # ...
          amoswap.w.rl x0, x0, (a0) # Release lock by storing 0.
          sd           x3, (a3)     # Random unrelated store
          ld           x4, (a4)     # Random unrelated load
  \end{verbatim}
  \caption{A spinlock with atomics}
  \label{fig:litmus:spinlock_atomics}
\end{figure}

Consider Figure~\ref{fig:litmus:spinlock_atomics}:
Because this example uses {\tt .aq}, the loads and stores in the critical section are guaranteed to appear in the global memory order after the {\tt amoswap} used to acquire the lock.  However, assuming {\tt a0}, {\tt a1}, and {\tt a2} point to different memory locations, the loads and stores in the critical section may or may not appear after the ``random unrelated load'' at the beginning of the example in the global memory order.

\begin{figure}[h!]
  \centering\small
  \begin{verbatim}
          sd           x1, (a1)     # Random unrelated store
          ld           x2, (a2)     # Random unrelated load
          li           t0, 1        # Initialize swap value.
      again:
          amoswap.w    t0, t0, (a0) # Attempt to acquire lock.
          fence        r, rw        # Enforce "acquire" memory ordering
          bnez         t0, again    # Retry if held.
          # ...
          # Critical section.
          # ...
          fence        rw, w        # Enforce "release" memory ordering
          amoswap.w    x0, x0, (a0) # Release lock by storing 0.
          sd           x3, (a3)     # Random unrelated store
          ld           x4, (a4)     # Random unrelated load
  \end{verbatim}
  \caption{A spinlock with fences}
  \label{fig:litmus:spinlock_fences}
\end{figure}

Now, consider the alternative in Figure~\ref{fig:litmus:spinlock_fences}.
In this case, even though the {\tt amoswap} does not enforce ordering with an {\tt .aq} bit, the fence nevertheless enforces that the acquire {\tt amoswap} appears earlier in the global memory order than all loads and stores in the critical section.
Note, however, that in this case, the fence also enforces additional orderings: it also requires that the ``random unrelated load'' at the start of the program appears also appears earlier in the global memory order than the loads and stores of the critical section.  (This particular fence does not, however, enforce any ordering with respect to the ``random unrelated store'' at the start of the snippet.)
In this way, fence-enforced orderings are slightly coarser than orderings enforced by {\tt.aq}.

Release orderings work exactly the same as acquire orderings, just in the opposite direction.  Release semantics require all loads and stores in the critical section to appear before the lock-releasing store (here, an {\tt amoswap}) in the global memory order.  This can be enforced using the {\tt .rl} bit or with a {\tt fence rw,w} operations.  Likewise, the ordering between the loads and stores in the critical section and the ``random unrelated store'' at the end of the code snippet is enforced only by the {\tt fence rw,w} in the second example, not by the {\tt .rl} in the first example.

With RCpc annotations alone, store-release-to-load-acquire ordering is not enforced.  This facilitates the porting of code written under the TSO and/or RCpc memory models.  C/C++ falls into this category as well.  See Section~\ref{sec:porting} for details.
To enforce store-release-to-load-acquire ordering, use store-release-RCsc and load-acquire-RCsc operations, so that PPO rule \ref{ppo:strongacqrel} applies.
The use of only store-release-RCsc and load-acquire-RCsc operations implies sequential consistency, as the combination of PPO rules \ref{ppo:acquire}--\ref{ppo:strongacqrel} implies that all RCsc accesses will respect program order.
However, this RCsc store-release-to-load-acquire ordering only applies when both of the operations in question have an RCsc annotation (or stronger).

\subsection{Atomics and LR/SCs (Rule~\ref{ppo:amoforward}, Atomicity Axiom)}
\begin{tabular}{p{2cm}|p{12cm}}
  Rule \ref{ppo:amoforward}: & \ppoamoforward \\
  Atomicity axiom: & \atomicityaxiom
\end{tabular}

The RISC-V architecture decouples the notion of atomicity from the notion of ordering.  Unlike architectures such as TSO, RISC-V atomics under RVWMO do not impose any ordering requirements by default.  Ordering semantics are only guaranteed by the PPO rules that otherwise apply.

RISC-V contains two types of atomics: AMOs and LR/SC pairs.
These conceptually behave differently, in the following way.
LR/SC behave as if the old value is brought up to the core, modified, and written back to memory, all while the reservation is held.
AMOs on the other hand conceptually behave as if they are performed directly in memory.
AMOs are therefore inherently atomic, while LR/SC pairs are atomic in the sense that the memory location in question will not be modified by another hart during the time the original hart holds the conceptual reservation.

\begin{verbbox}
(a) lr.d t0, 0(a0)
(b) sd   t1, 0(a0)
(c) sc.d t2, 0(a0)
\end{verbbox}
\begin{figure}[h!]
  \centering\small
  \theverbbox
  \caption{Store-conditional (c) may succeed on some implementations}
  \label{fig:litmus:lrsdsc}
\end{figure}

The atomicity axiom does not forbid loads from being interleaved between the paired operations in program order or in the global memory order, nor does it forbid stores from the same hart from appearing between the paired operations in either program order or in the global memory order.
For example, the sequence in Figure~\ref{fig:litmus:lrsdsc} is legal, and the {\tt sc} may (but is not guaranteed to) succeed.
By preserved program order rule \ref{ppo:->st}, the program order of the three operations must be maintained in the global memory order.  This does not violate the atomicity axiom, because the intervening non-conditional store is from the same hart as the paired load-reserved and store-conditional instructions.
This way, a memory system that tracks all memory accesses beyond L1 at cache line granularity will not be forced to fail a store conditional instruction that happens to (falsely) share another portion of the same cache line as the memory held by the reservation.

Rule~\ref{ppo:amoforward} simply states that a value cannot be returned from an AMO to a subsequent load until the AMO has performed globally.
This follows naturally from the intuition that AMOs are performed directly in memory, except possibly for the case of {\tt amoswap}.
Rule~\ref{ppo:amoforward} states that hardware may not (non-speculatively) forward the value being stored by the {\tt amoswap} to a subsequent load, even though that store value is not actually semantically dependent on the previous value in memory, as is the case for the other AMOs.

\subsection{Dependencies (Rules~\ref{ppo:addr}--\ref{ppo:ctrl})}
\label{sec:depspart1}
\begin{tabular}{p{1cm}|p{12cm}}
  & Rule \ref{ppo:addr}: \ppoaddr \\
  & Rule \ref{ppo:data}: \ppodata \\
  & Rule \ref{ppo:ctrl}: \ppoctrl \\
\end{tabular}

Dependencies from a load to a later memory operation in the same hart are respected by the RVWMO memory model.
The Alpha memory model was notable for choosing {\em not} to enforce the ordering of such dependencies, but most modern hardware and software memory models consider allowing dependent instructions to be reordered too confusing and counterintuitive.
Furthermore, modern code sometimes intentionally uses such dependencies as a particularly lightweight ordering enforcement mechanism.

The terms in Section~\ref{sec:deps} work as follows.
Instructions are said to carry dependencies from their source register(s) to their destination register(s) whenever the value written into each destination register is a function of the source register(s).
For most instructions, this means that the destination register(s) carry a dependency from all source register(s).
However, there are a few notable exceptions.
In the case of memory instructions, the value written into the destination register ultimately comes from the memory system rather than from the source register(s) directly, and so this breaks the chain of dependencies carried from the source register(s).
In the case of indirect jumps, the value written into the destination register comes from the current {\tt pc} (which is never considered a source register by the memory model), and so likewise, indirect jumps do not carry a dependency from the source register(s) to the destination register.

Lastly, the notion of accumulating into a destination register rather than writing into it reflects the behavior of CSRs such as {\tt fflags}.
In particular, an accumulation into a register does not clobber any previous writes or accumulations into the same register.
For example, in Figure~\ref{fig:litmus:fflags}, (c) has a syntactic dependency on both (a) and (b).
\begin{verbbox}
(a) fadd  f3,f1,f2
(b) fadd  f6,f4,f5
(c) csrrs a0,fflags,x0
\end{verbbox}
\begin{figure}
  \centering\small
  \theverbbox
  \caption{(c) has a syntactic dependency on both (a) and (b) via {\tt fflags}, a destination register which both (a) and (b) implicitly accumulate into}
  \label{fig:litmus:fflags}
\end{figure}

Like other modern memory models, the RVWMO memory model uses syntactic rather than semantic dependencies.
In other words, this definition depends on the identities of the
registers being accessed by different instructions, not the actual
contents of those registers.  This means that an address, control, or
data dependency must be enforced even if the calculation could seemingly
be ``optimized away''.
This choice ensures that RVWMO remains compatible with programmers that use these false syntactic dependencies intentionally to form a lightweight type of ordering mechanism.

For example, there is a syntactic address
dependency from the first instruction to the last instruction in the
Figure~\ref{fig:litmus:address}, even though {\tt a1} XOR {\tt a1} is zero and
hence has no effect on the address accessed by the second load.
\begin{verbbox}
ld  a1,0(s0)
xor a2,a1,a1
add s1,s1,a2
ld  a5,0(s1)
\end{verbbox}
\begin{figure}[h!]
  \centering\small
  \theverbbox
  \caption{A syntactic address dependency}
  \label{fig:litmus:address}
\end{figure}

The benefit of using dependencies as a lightweight synchronization mechanism is that the ordering enforcement requirement is limited only to the specific two instructions in question.
Other non-dependent instructions may be freely-reordered by aggressive implementations.
One alternative would be to use a load-acquire, but this would enforce ordering for the first load with respect to {\em all} subsequent instructions.
Another would be to use a {\tt fence r,r}, but this would include all previous and all subsequent loads, making this option more expensive.

Control dependencies behave differently from address and data dependencies in the sense that a control dependency always extends to all instructions following the original target in program order.
Consider Figure~\ref{fig:litmus:control1}: the instruction at {\tt next} will always execute, but it nevertheless still has control dependency from the first instruction.
\begin{verbbox}
      lw  x1,0(x2)
      bne x1,x0,NEXT
      sw  x3,0(x4)
next: sw  x5,0(x6)
\end{verbbox}
\begin{figure}[h!]
  \centering\small
  \theverbbox
  \caption{A syntactic control dependency}
  \label{fig:litmus:control1}
\end{figure}

\begin{verbbox}
        lw  x1,0(x2)
        bne x1,x0,NEXT
  next: sw  x3,0(x4)
\end{verbbox}
\begin{figure}[h!]
  \centering\small
  \theverbbox
  \caption{Another syntactic control dependency}
  \label{fig:litmus:control2}
\end{figure}

Likewise, consider Figure~\ref{fig:litmus:control2}.
Even though both branch outcomes have the same target, there is still a control dependency from the first instruction in this snippet to the last.
This definition of control dependency is subtly stronger than what might be seen in other contexts (e.g., C++), but it conforms with standard definitions of control dependencies in the literature.

Notably, PPO rules \ref{ppo:addr}--\ref{ppo:ctrl} are also intentionally designed to respect dependencies which originate from the output of a successful store conditional instruction.
In general, an {\tt sc} instruction will be followed by a branch checking whether the outcome was successful; this implies that there will be a control dependency from the store operation generated by the {\tt sc} instruction to any memory operations following the branch.
PPO rule~\ref{ppo:ctrl} in turn implies that any subsequent store operations will appear later in the global memory order than the store operation generated by the {\tt sc}.

\begin{figure}[h!]
  \center
  {
    \tt\small
    \begin{tabular}{cl||cl}
    \multicolumn{2}{c}{Hart 0} & \multicolumn{2}{c}{Hart 1} \\
    \hline
      (a) & ld a0,0(s0)    & (e) & ld a3,0(s2) \\
      (b) & lr a1,0(s1)    & (f) & sd a3,0(s0) \\
      (c) & sc a2,a0,0(s1) &                    \\
      (d) & sd a2,0(s2)    &                    \\
    \end{tabular}
  }
  ~~~~
  \diagram
  \caption{A variant of the LB litmus test}
  \label{fig:litmus:successdeps}
\end{figure}

In addition, the choice to respect dependencies originating at store-conditional instructions ensures that certain out-of-thin-air-like behaviors will be prevented.
Consider Figure~\ref{fig:litmus:successdeps}.
Suppose a hypothetical implementation could occasionally make some early guarantee that a store-conditional operation will succeed.
In this case, (c) could return 0 to {\tt a2} early (before actually executing), allowing the sequence (d), (e), (f), (a), and then (b) to execute, and then (c) might execute (successfully) only at that point.
This would imply that (c) writes its own success value to {\tt 0(s1)}!
Fortunately, this situation and others like it are prevented by the fact that RVWMO respects dependencies originating at the stores generated by successful {\tt sc} instructions.

\subsection{Same-Address Load-Load Ordering (Rule~\ref{ppo:rdw})}
\label{sec:ppo:rdw}
\begin{tabular}{p{1cm}|p{12cm}}
  & Rule \ref{ppo:rdw}: \ppordw \\
\end{tabular}

In contrast to same-address orderings ending in a store, same-address load-load ordering requirements are very subtle.

The basic requirement is that a younger load must not return a value which is older than a value returned by an older load in the same hart to the same address.  This is often known as ``CoRR'' (Coherence for Read-Read pairs), or as part of a broader ``coherence'' or ``sequential consistency per location'' requirement.
Some architectures in the past have relaxed same-address load-load ordering, but in hindsight this is generally considered to complicate the programming model too much, and so RVWMO requires CoRR ordering to be enforced.
However, because the global memory order corresponds to the order in which loads perform rather than the ordering of the values being returned, capturing CoRR requirements in terms of the global memory order requires a bit of indirection.

\begin{figure}[h!]
  \center
  {
    \tt\small
    \begin{tabular}{cl||cl}
    \multicolumn{2}{c}{Hart 0} & \multicolumn{2}{c}{Hart 1} \\
    \hline
          & li t1, 1    &     & li~ t2, 2    \\
      (a) & sw t1,0(s0) & (d) & lw~ a0,0(s1) \\
      (b) & fence w, w  & (e) & sw~ t2,0(s1) \\
      (c) & sw t1,0(s1) & (f) & lw~ a1,0(s1) \\
          &             & (g) & xor t3,a1,a1 \\
          &             & (h) & add s0,s0,t3 \\
          &             & (i) & lw~ a2,0(s0) \\
    \end{tabular}
  }
  ~~~~
  \diagram
  \caption{Litmus test MP+FENCE+fri-rfi-addr}
  \label{fig:litmus:frirfi}
\end{figure}

Consider the litmus test of Figure~\ref{fig:litmus:frirfi}, which is one particular instance of the more general ``fri-rfi'' pattern.
The term ``fri-rfi'' refers to the sequence (d),(e),(f): (d) ``from-reads'' (i.e., reads from an earlier write than) (e) which is the same hart, and (f) reads from (e) which is in the same hart.

From a microarchitectural perspective, outcome {\tt a0=1, a1=2, a2=0} is legal (as are various other less subtle outcomes).  Intuitively, the following would produce the outcome in question:
\begin{itemize}
  \item (a), (b), (c) execute
  \item (d) stalls (for whatever reason; perhaps it's stalled waiting for some other preceding instruction)
  \item (e) executes and enters the store buffer
  \item (f) forwards from (e) in the store buffer
  \item (g), (h), and (i) execute
  \item (d) unstalls and executes
  \item (e) drains from the store buffer to memory
\end{itemize}
This corresponds to a global memory order of (e),(f),(i),(a),(c),(d).
Note that even though (f) performs before (d), the value returned by (f) is newer than the value returned by (d).
Therefore, this execution is legal and does not violate the CoRR requirements even though (f) appears before (d) in global memory order.

Likewise, if two back-to-back loads return the values written by the same store, then they may also appear out-of-order in the global memory order without violating CoRR.  Note that this is not the same as saying that the two loads return the same value, since two different stores may write the same value.   Consider the litmus test of Figure~\ref{fig:litmus:rsw}:

\begin{figure}[h!]
  \centering
  {
    \tt\small
    \begin{tabular}{cl||cl}
    \multicolumn{2}{c}{Hart 0} & \multicolumn{2}{c}{Hart 1} \\
    \hline
          & li t1, 1    & (d) & lw~ a0,0(s1) \\
      (a) & sw t1,0(s0) & (e) & xor t2,a0,a0 \\
      (b) & fence w, w  & (f) & add s2,s2,t2 \\
      (c) & sw t1,0(s1) & (g) & lw~ a1,0(s2) \\
          &             & (h) & lw~ a2,0(s2) \\
          &             & (i) & xor t3,a2,a2 \\
          &             & (j) & add s0,s0,t3 \\
          &             & (k) & lw~ a3,0(s0) \\
    \end{tabular}
  }
  ~~~~
  \diagram
  \caption{Litmus test RSW}
  \label{fig:litmus:rsw}
\end{figure}

The outcome {\tt a0=1,a1=a2,a3=0} can be observed by allowing (g) and (h) to be reordered.  This might be done speculatively, and the speculation can be justified by the microarchitecture (e.g., by snooping for cache invalidations and finding none) because replaying (h) after (g) would return the value written by the same store anyway.
Hence assuming {\tt a1=a2}, (g) and (h) can be reordered.
The global memory order corresponding to this execution would be (h),(k),(a),(c),(d),(g).

Executions of the above test in which {\tt a1} does not equal {\tt a2} do in fact require that (g) appears before (h) in the global memory order.
Allowing (h) to appear before (g) in the global memory order would in fact result in a violation of CoRR, because then (h) would return an older value than that returned by (g).
Therefore, PPO rule~\ref{ppo:rdw} forbids this CoRR violation from occurring.
As such, PPO rule~\ref{ppo:rdw} strikes a careful balance between enforcing CoRR in all cases while simultaneously being weak enough to permit ``RSW'' and ``fri-rfi'' patterns that commonly appear in real microarchitectures.


\subsection{Pipeline Dependency Artifacts (Rules~\ref{ppo:ld->st->ld}--\ref{ppo:addrpo})}
\label{sec:ppopipeline}
\begin{tabular}{p{1cm}|p{12cm}}
  & Rule \ref{ppo:ld->st->ld}: \ppoldstld \\
  & Rule \ref{ppo:addrpo}: \ppoaddrpo \\
%  & Rule \ref{ppo:ctrlcfence}: \ppoctrlcfence \\
%  & Rule \ref{ppo:addrpocfence}: \ppoaddrpocfence \\
\end{tabular}

These four ``compound dependency'' rules reflect behaviors of almost all real processor pipeline implementations, and they are added into the model explicitly to simplify the definition of the formal operational memory model and to improve compatibility with known patterns on other architectures.

\begin{figure}[h!]
  \centering
  {
    \tt\small
    \begin{tabular}{cl}
      (a) & lw a0, 0(s0)   \\
      (b) & sw a0, 0(s1)   \\
      (c) & lw a1, 0(s1)   \\
    \end{tabular}
  }
  ~~~~
  \diagram
  \caption{Because of the data dependency from (a) to (b), (a) also precedes (c)}
  \label{fig:litmus:addrdatarfi}
\end{figure}

Rule~\ref{ppo:ld->st->ld} states that a load forward from a store until the address and data for that store are known.
Consider Figure~\ref{fig:litmus:addrdatarfi}:
(c) cannot be executed until the data for (b) has been resolved, because (c) must return the value written by (b) (or by something even later in the global memory order).  Therefore, (c) will never execute before (a) has executed.

\begin{figure}[h!]
  \centering
  {
    \tt\small
    \begin{tabular}{cl}
          & li t1, 1       \\
      (a) & lw a0, 0(s0)   \\
      (b) & sw a0, 0(s1)   \\
          & sw t1, 0(s1)   \\
      (c) & lw a1, 0(s1)   \\
    \end{tabular}
  }
  ~~~~
  \diagram
  \caption{Because of the extra store between (b) and (c), (a) no longer necessarily precedes (c)}
  \label{fig:litmus:addrdatarfi_no}
\end{figure}

If there were another store to the same address in between (b) and (c), as in Figure~\ref{fig:litmus:addrdatarfi_no}, then (c) would no longer dependent on the data of (b) being resolved, and hence the dependency of (c) on (a), which produces the data for (b), would be broken.

Rule~\ref{ppo:addrpo} makes a similar observation to the previous rule: a store cannot be performed at memory until all previous loads which might access the same address have themselves been performed.
Such a load must appear to execute before the store, but it cannot do so if the store were to overwrite the value in memory before the load had a chance to read the old value.

\begin{figure}[h!]
  \centering
  {
    \tt\small
    \begin{tabular}{cl}
        & li t1, 1       \\
    (a) & lw a0, 0(s0)   \\
    (b) & lw a1, 0(a0)   \\
    (c) & sw t1, 0(s1)   \\
    \end{tabular}
  }
  ~~~~
  \diagram
  \caption{Because of the address dependency from (a) to (b), (a) also precedes (c)}
  \label{fig:litmus:addrpo}
\end{figure}

Consider Figure~\ref{fig:litmus:addrpo}:
(c) cannot be executed until the address for (b) is resolved, because it may turn out that the addresses match; i.e., that {\tt a0=s1}.  Therefore, (c) cannot be sent to memory before (a) has executed and confirmed whether the addresses to indeed overlap.

\subsection{Progress Axiom}
\label{sec:ppopipeline}
\begin{tabular}{p{1cm}|p{12cm}}
  & \progressaxiom \\
\end{tabular}

The progress axiom ensures a minimal forward progress guarantee.
It ensures that stores from one hart will eventually be made visible to other harts in the system in a finite amount of time, and that loads from other harts will eventually be able to read those values (or successors thereof).
Without this rule, it would be legal, for example, for a spinlock to spin infinitely on a value, even with a store from another hart waiting to unlock the spinlock.

The progress axiom is intended not to impose any other notion of fairness or quality of service onto the harts in a RISC-V implementation.
Any stronger notions of fairness are up to the rest of the ISA and/or up to the platform to define and implement.

The forward progress axiom will in almost all cases be naturally satisfied by any standard cache coherence protocol.
Implementations with non-coherent caches may have to provide some other mechanism to ensure the eventual visibility of all stores (or successors thereof) to all harts.


\section{FENCE.I, SFENCE.VMA, and I/O Fences}

\textcolor{red}{DL: I stopped here...}

In this section, we provide an informal description of how the {\tt fence.i}, {\tt sfence.vma}, and I/O fences interact with the memory model.

Instruction fetches and address translation operations (where applicable) follow the RISC-V memory model as well as the rules below.
\begin{itemize}
  \item {\tt fence.i}: Conceptually, {\tt fence.i} ensures that the fetch of each instruction following the {\tt fence.i} in program order appears later in the global memory order than all stores prior to the {\tt fence.i} in program order.
    That in turn means that instruction caches which hardware does not keep coherent with normal memory must be flushed when a {\tt fence.i} instruction is executed.
    %({\tt fence.i} is also used form the patterns of Section~\ref{sec:ppopipeline}.)
  \item {\tt sfence.vma}: Conceptually, the instruction fetch and address translation operations of each instruction following the {\tt sfence.vma} in program order appears later in the global memory order than all stores prior to the {\tt sfence.vma} in program order.
    This implies that stale entries in the local hart's TLBs must be invalidated.
  \item Conceptually, updates to the page table made by a hardware page table walker form a paired atomic read-modify-write operation subject to the rules of the atomicity axiom
\end{itemize}

\subsection{Coherence and Cacheability}

The RISC-V ISA defines Physical Memory Attributes (PMAs) which specify, among other things, whether portions of the address space are coherent and/or cacheable.
See the privileged spec for the complete details.
Here, we simply discuss how the various details in each PMA relate to the memory model:

\begin{itemize}
  \item Main memory vs.\@ I/O, and I/O memory ordering PMAs: the memory model as defined applies to main memory regions.  I/O ordering is discussed below.
  \item Supported access types and atomicity PMAs: the memory model is simply applied on top of whatever primitives each region supports.
  \item Cacheability PMAs: the cacheability PMAs do not affect the memory model.  Non-cacheable regions may have more restrictive behavior than cacheable regions, but the set of allowed behaviors does not change regardless.
  \item Coherence PMAs: The memory consistency model for memory regions marked as non-coherent in PMAs is currently platform-specific: the load-value axiom, the atomicity axiom, and the progress axiom all may be violated with non-coherent memory.  Note however that coherent memory does not require a hardware cache coherence protocol.  The RISC-V privileged specification suggests that hardware-incoherent regions of main memory are discouraged, but the memory model is compatible with hardware coherence, software coherence, implicit coherence due to read-only memory, implicit coherence due to only one agent having access, or otherwise.
  \item Idempotency PMAs: Idempotency PMAs are used to specify memory regions for which loads and/or stores may have side effects, and this in turn is used by the microarchitecture to determine, e.g., whether prefetches are legal.  This distinction does not affect the memory model.
\end{itemize}


\subsection{I/O Ordering}

\textcolor{red}{FIXME: AMOs to I/O will generally have to follow a stronger notion of atomicity.}

For I/O, the load value axiom and atomicity axiom in general do not apply, as both reads and writes might have device-specific side effects.
The preserved program order rules do not generally apply to I/O either.
Instead, we informally say that memory access $a$ precedes memory access $b$ in global memory order if $a$ precedes $b$ in program order and one or more of the following holds:
\begin{enumerate}
  \item $a$ and $b$ are accesses to overlapping addresses in an I/O region
  \item $a$ and $b$ are accesses to the same strongly-ordered I/O region
  \item $a$ and $b$ are accesses to I/O regions, and the channel associated with the I/O region accessed by either $a$ or $b$ is channel 1
  \item $a$ and $b$ are accesses to I/O regions associated with the same channel (except for channel 0)
  \item $a$ and $b$ are separated in program order by a FENCE, $a$ is in the predecessor set of the FENCE, and $b$ is in the successor set of the FENCE.  The predecessor and successor sets include the sets described by all eight FENCE bits {\tt .pr}, {\tt .pw}, {\tt .pi}, {\tt .po}, {\tt .sr}, {\tt .sw}, {\tt .si}, and {\tt .so}.
  \item $a$ and $b$ are accesses to I/O regions, and $a$ has {\tt .aq} set
  \item $a$ and $b$ are accesses to I/O regions, and $b$ has {\tt .rl} set
  \item $a$ and $b$ are accesses to I/O regions, and $a$ and $b$ both have {\tt .aq} and {\tt .rl} set
  \item $a$ and $b$ are accesses to I/O regions, and $a$ is an AMO that has {\tt .aq} and {\tt .rl} set
  \item $a$ and $b$ are accesses to I/O regions, and $b$ is an AMO that has {\tt .aq} and {\tt .rl} set
\end{enumerate}

As described above, accesses to I/O memory require stronger synchronization that what is enforced by the RVWMO PPO rules.
For such cases, {\tt FENCE} operations with {\tt .pi}, {\tt .po}, {\tt .si}, and/or {\tt .so} are needed.
For example, to enforce ordering between a write to normal memory and an MMIO write to a device register, a {\tt FENCE w,o} or stronger is needed.
Even {\tt .aq} and {\tt .rl} do not enforce ordering between normal memory accesses and accesses to I/O memory.
When a fence is in fact used, implementations must assume that the device may attempt to access memory immediately after receiving the MMIO signal, and subsequent memory accesses from that device to memory must observe the effects of all accesses ordered prior to that MMIO operation.

\begin{verbbox}
  sd t0, 0(a0)
  fence w,o
  sd a0, 0(a1)
\end{verbbox}
\begin{figure}[h!]
  \centering\small
  \theverbbox
  \caption{Ordering memory and I/O accesses}
  \label{fig:litmus:wo}
\end{figure}

In other words, in Figure~\ref{fig:litmus:wo}, suppose {\tt 0(a0)} is in normal memory and {\tt 0(a1)} is the address of a device register in I/O memory.
If the device accesses {\tt 0(a0)} upon receiving the MMIO write, then that load must conceptually appear after the first store to {\tt 0(a0)} according to the rules of the RVWMO memory model.
In some implementations, the only way to ensure this will be to require that the first store does in fact complete before the MMIO write is issued.
Other implementations may find ways to be more aggressive, while others still may not need to do anything different at all for I/O and normal memory accesses.
Nevertheless, the RVWMO memory model does not distinguish between these options; it simply provides an implementation-agnostic mechanism to specify the orderings that must be enforced.

Many architectures include separate notions of ``ordering'' and ``completion'' fences, especially as it relates to I/O (as opposed to normal memory).
Ordering fences simply ensure that memory operations stay in order, while completion fences ensure that predecessor accesses have all completed before any successors are made visible.
RISC-V does not explicitly distinguish between ordering and completion fences.
Instead, this distinction is simply inferred from different uses of the FENCE bits.

For implementations that conform to the RISC-V Unix Platform Specification, I/O devices, DMA operations, etc.\@ are required to access memory coherently and via strongly-ordered I/O channels.
Therefore, accesses to normal memory regions that are shared with I/O devices can also use the standard synchronization mechanisms.
Implementations which do not conform to the Unix Platform Specification and/or in which devices do not access memory coherently will need to use platform-specific mechanisms (such as cache flushes) to enforce coherency.

I/O regions in the address space should be considered non-cacheable regions in the PMAs for those regions.  Such regions can be considered coherent by the PMA if they are not cached by any agent.

The ordering guarantees in this section may not apply beyond a platform-specific boundary between the RISC-V cores and the device.  In particular, I/O accesses sent across an external bus (e.g., PCIe) may be reordered before they reach their ultimate destination.  Ordering must be enforced in such situations according to the platform-specific rules of those external devices and buses.

\section{Code Examples}
\label{sec:mmcode}

\subsection{Compare and Swap}
An example
using {\tt lr}/{\tt sc} to implement a compare-and-swap function is shown in
Figure~\ref{cas}.  If inlined, compare-and-swap functionality need
only take three instructions.

\begin{figure}[h!]
\begin{center}
\begin{verbatim}
        # a0 holds address of memory location 
        # a1 holds expected value
        # a2 holds desired value
        # a0 holds return value, 0 if successful, !0 otherwise
    cas:
        lr.w t0, (a0)        # Load original value.
        bne t0, a1, fail     # Doesn't match, so fail.
        sc.w a0, a2, (a0)    # Try to update.
        jr ra                # Return.
    fail:
        li a0, 1             # Set return to failure.
        jr ra                # Return.
\end{verbatim}
\end{center}
  \caption{Sample code for compare-and-swap function using {\tt lr}/{\tt sc}.}
\label{cas}
\end{figure}

\subsection{Spinlocks}
\label{sec:spinlock}

An example code sequence for a critical section guarded by a
test-and-set spinlock is shown in Figure~\ref{critical}.  Note the
first AMO is marked {\tt .aq} to order the lock acquisition before the
critical section, and the second AMO is marked {\tt .rl} to order
the critical section before the lock relinquishment.

\begin{figure}[h!]
\begin{center}
\begin{verbatim}
        li           t0, 1        # Initialize swap value.
    again:
        amoswap.w.aq t0, t0, (a0) # Attempt to acquire lock.
        bnez         t0, again    # Retry if held.
        # ...
        # Critical section.
        # ...
        amoswap.w.rl x0, x0, (a0) # Release lock by storing 0.
\end{verbatim}
\end{center}
\caption{Sample code for mutual exclusion.  {\tt a0} contains the address of the lock.}
\label{critical}
\end{figure}

\section{Code Porting Guidelines}
\label{sec:porting}

Normal x86 loads and stores are all inherently acquire and release operations: TSO enforces all load-load, load-store, and store-store ordering by default.
All TSO loads must be mapped onto {\tt l\{b|h|w|d\}; fence r,rw}, and all TSO stores must either be mapped onto {\tt amoswap.rl x0} or onto {\tt fence rw,w; s\{b|h|w|d\}}.
Alternatively, TSO loads and stores can be mapped onto {\tt l\{b|h|w|d\}.aq} and {\tt s\{b|h|w|d\}.rl} assembler pseudoinstructions to facilitate forwards compatibility in case such instructions are added to the ISA one day.
However, in the meantime, the assembler will generate the same fence-based and/or {\tt amoswap}-based versions for these pseudoinstructions.
%However, the more correct solution to porting code from x86-TSO (which is generally overly-constrained at the assembly level compared to DRF software requirements) is to rewrite the algorithm to determine which orderings the original algorithm actually required, and then to re-code the algorithm in terms of the RVWMO memory model.
x86 atomics using the LOCK prefix are all sequentially consistent and when ported naively to RISC-V must be marked as {\tt .aqrl}.

A Power {\tt sync}/{\tt hwsync} fence, an ARM {\tt dmb} fence, and an x86 {\tt mfence} are all equivalent to a RISC-V {\tt fence rw,rw}.
Power {\tt isync} and ARM {\tt isb} map to RISC-V {\tt fence r,r}.
A Power {\tt lwsync} map onto {\tt fence.tso}, or onto {\tt fence rw,rw} when {\tt fence.tso} is not available.
ARM {\tt dmb ld} and {\tt dmb st} fences map to RISC-V {\tt fence r,rw} and {\tt fence w,w}, respectively.

A direct mapping of ARMv8 atomics that maps unordered instructions to unordered instructions, RCpc instructions to RCpc instructions, and RCsc instructions to RCsc instructions is likely to work in the majority of cases.
Mapping even unordered load-reserved instructions onto {\tt lr.aq} (particularly for LR/SC pairs without internal data dependencies) is an even safer bet, as this ensures C/C++ release sequences will be respected.
However, due to a subtle mismatch between the two models, strict theoretical compatibility with the ARMv8 memory model requires that a naive mapping translate all ARMv8 store conditional and load-acquire operations map onto RISC-V RCsc operations.
Any atomics which are naively ported into RCsc operations may revert back to the straightforward mapping if the programmer can verify that the code is not relying on an ordering from the store-conditional to the load-acquire (as this is not common).

%ARMv8 solves the C/C++ release sequence problem of Section~\ref{sec:acqrel} through a rule that is different from rule~\ref{ppo:loadtoacq}.
%Therefore, strict formal compatibility requires naively-ported ARMv8 load-acquire operations to be preceded by a {\tt fence w,r,[addr]} or stronger.
%The naive translations of ARM {\tt ldapr}, {\tt ldar}, and {\tt stlr} are therefore ``{\tt fence w,r,[addr]; amoor.aq rd,[addr],x0}'', ``{\tt fence w,r,[addr]; amoor.aq.rl rd,[addr],x0}'' and ``{\tt amoswap.aq.rl} with {\tt rd=x0}'', respectively.
%In general the extra fence would not have been necessary if the original source were recompiled to RISC-V natively, because the RVWMO memory model already solves the same underlying problem just in a different way.
%Naive ports may choose whether to stick to a strict naive port or to assume that the (cheaper) mapping without the fence is more than likely sufficient.

The Linux fences {\tt smp\_mb()}, {\tt smp\_wmb()}, {\tt smp\_rmb()} map onto {\tt fence rw,rw}, {\tt fence w,w}, and {\tt fence r,r}, respectively.  The fence {\tt smp\_read\_barrier\_depends()} map to a no-op due to preserved program order rules \ref{ppo:addr}--\ref{ppo:ctrl}.
The Linux fences {\tt dma\_rmb()} and {\tt dma\_wmb()} map onto {\tt fence r,r} and {\tt fence w,w}, respectively, since the RISC-V Unix Platform requires coherent DMA.
The Linux fences {\tt rmb()}, {\tt wmb()}, and {\tt mb()} map onto {\tt fence ri,ri}, {\tt fence wo,wo}, and {\tt fence rwio,rwio}, respectively.

%\begin{table}[h!]
%  \begin{tabular}{|l|l|l|}
%    \hline
%    C/C++ Construct                            & Base ISA Mapping & `A' Extension Mapping \\
%    \hline
%    \hline
%    Non-atomic load                            & \multicolumn{2}{l|}{\tt ld}               \\
%    \hline
%    \tt atomic\_load(memory\_order\_relaxed)   & \multicolumn{2}{l|}{\tt ld}               \\
%    \hline
%    %\tt atomic\_load(memory\_order\_consume)   & \multicolumn{2}{l|}{\tt ld; fence r,rw}   \\
%    %\hline
%    \tt atomic\_load(memory\_order\_acquire)   & \tt fence r,r,[addr]; & \tt ld.aq   \\
%                                               & \tt ld; fence r,rw    &                       \\
%    \hline
%    \tt atomic\_load(memory\_order\_seq\_cst)  & \tt fence rw,rw; ld; & \tt ld.aq.rl rs2=x0   \\
%                                               & \tt fence r,rw       &                       \\
%    \hline
%    \hline
%    Non-atomic store                           & \multicolumn{2}{l|}{\tt sd}               \\
%    \hline
%    \tt atomic\_store(memory\_order\_relaxed)  & \multicolumn{2}{l|}{\tt sd}               \\
%    \hline
%    \tt atomic\_store(memory\_order\_release)  & \tt fence rw,w; sd  & \tt sd.rl x0   \\
%    \hline
%    \tt atomic\_store(memory\_order\_seq\_cst) & \tt fence rw,rw; sd & \tt sd.aq.rl x0  \\
%    \hline
%    \hline
%    \tt atomic\_thread\_fence(memory\_order\_acquire)  & \multicolumn{2}{l|}{\tt fence r,rw} \\
%    \hline
%    \tt atomic\_thread\_fence(memory\_order\_release)  & \multicolumn{2}{l|}{\tt fence rw,w} \\
%    \hline
%    \tt atomic\_thread\_fence(memory\_order\_acq\_rel) & \multicolumn{2}{l|}{{\tt fence rw,rw~} or {~\tt fence rw,w; fence r,rw}} \\
%    \hline
%    \tt atomic\_thread\_fence(memory\_order\_seq\_cst) & \multicolumn{2}{l|}{\tt fence rw,rw} \\
%    \hline
%  \end{tabular}
%  \caption{Mappings from C/C++ primitives to RISC-V primitives.  The atomics mapping is preferable where available.}
%  \label{tab:mappings}
%\end{table}

\begin{table}[h!]
  \begin{tabular}{|l|l|}
    \hline
    C/C++ Construct                            & RVWMO Mapping \\
    \hline
    \hline
    Non-atomic load                            & \tt l\{b|h|w|d\}               \\
    \hline
    \tt atomic\_load(memory\_order\_relaxed)   & \tt l\{b|h|w|d\}               \\
    \hline
    %\tt atomic\_load(memory\_order\_consume)   & \multicolumn{2}{l|}{\tt l\{b|h|w|d\}; fence r,rw}   \\
    %\hline
    \tt atomic\_load(memory\_order\_acquire)   & \tt l\{b|h|w|d\}; fence r,rw    \\
    \hline
    \tt atomic\_load(memory\_order\_seq\_cst)  & \tt fence rw,rw; l\{b|h|w|d\}; fence r,rw       \\
    \hline
    \hline
    Non-atomic store                           & \tt s\{b|h|w|d\}               \\
    \hline
    \tt atomic\_store(memory\_order\_relaxed)  & \tt s\{b|h|w|d\}               \\
    \hline
    \tt atomic\_store(memory\_order\_release)  & \tt fence rw,w; s\{b|h|w|d\}  \\
    \hline
    \tt atomic\_store(memory\_order\_seq\_cst) & \tt fence rw,rw; s\{b|h|w|d\} \\
    \hline
    \hline
    \tt atomic\_thread\_fence(memory\_order\_acquire)  & \tt fence r,rw \\
    \hline
    \tt atomic\_thread\_fence(memory\_order\_release)  & \tt fence rw,w \\
    \hline
    \tt atomic\_thread\_fence(memory\_order\_acq\_rel) & {\tt fence.tso} \\
    \hline
    \tt atomic\_thread\_fence(memory\_order\_seq\_cst) & \tt fence rw,rw \\
    \hline
  \end{tabular}
  \caption{Mappings from C/C++ primitives to RISC-V primitives.} %  The atomics mapping is preferable where available.}
  \label{tab:mappings}
\end{table}

The C11/C++11 {\tt memory\_order\_*} primitives should be mapped as shown in Table~\ref{tab:mappings}.
The {\tt memory\_order\_release} mappings may use {\tt .rl} as an alternative.

\begin{table}[h!]
\centering
  \begin{tabular}{|l|l|}
    \hline
    Ordering Annotation & Fence-based Implementation \\
    \hline
    \tt l\{b|h|w|d|r\}.aq        & \tt l\{b|h|w|d|r\}; fence r,rw \\
    \hline
    \tt l\{b|h|w|d|r\}.aqrl      & \tt fence rw,rw; l\{b|h|w|d|r\}; fence r,rw \\
    \hline
    \tt s\{b|h|w|d|c\}.rl        & \tt fence rw,w; s\{b|h|w|d|c\} \\
    \hline
    \tt s\{b|h|w|d|c\}.aqrl      & \tt fence rw,w; s\{b|h|w|d|c\} \\
    \hline
    \tt amo<op>.aq               & \tt amo<op>; fence r,rw \\
    \hline
    \tt amo<op>.rl               & \tt fence rw,w; amo<op> \\
    \hline
    \tt amo<op>.aqrl             & \tt fence rw,rw; amo<op>; fence rw,rw \\
    \hline
  \end{tabular}
  \caption{Mappings from {\tt .aq} and/or {\tt .rl} to fence-based implementations.  An alternative mapping places a {\tt fence rw,rw} after the existing {\tt s\{b|h|w|d|c\}} mapping rather than at the front of the {\tt l\{b|h|w|d|r\}} mapping.}
  \label{tab:aqrltofence}
\end{table}

It is also safe to translate any {\tt .aq}, {\tt .rl}, or {\tt .aqrl} annotation into the fence-based snippets of Table~\ref{tab:aqrltofence}.
These can also be used as a legal implementation of {\tt l\{b|h|w|d\}} or {\tt s\{b|h|w|d\}} pseudoinstructions for as long as those instructions are not added to the ISA.


\section{Implementation Guidelines}

The RVWMO and RVTSO memory models by no means preclude microarchitectures from employing sophisticated speculation techniques or other forms of optimization in order to deliver higher performance.
The models also do not impose any requirement to use any one particular cache hierarchy, nor even to use a cache coherence protocol at all.
Instead, these models only specify the behaviors that can be exposed to software.
Microarchitectures are free to use any pipeline design, any coherent or non-coherent cache hierarchy, any on-chip interconnect, etc., as long as the design satisfy the memory model rules.
That said, to help people understand the actual implementations of the memory model, in this section we provide some guidelines below on how architects and programmers should interpret the models' rules.

Both RVWMO and RVTSO are multi-copy atomic (or ``other-multi-copy-atomic''): any store value which is visible to a hart other than the one that originally issued it must also be conceptually visible to all other harts in the system.
In other words, harts may forward from their own previous stores before those stores have become globally visible to all harts, but no early inter-hart forwarding is permitted.
Multi-copy atomicity may be enforced in a number of ways.
It might hold inherently due to the physical design of the caches and store buffers, it may be enforced via a single-writer/multiple-reader cache coherence protocol, or it might hold due to some other mechanism.

Although multi-copy atomicity does impose some restrictions on the microarchitecture, it is one of the key properties keeping the memory model from becoming extremely complicated.
For example, a hart may not legally forward a value from a neighbor hart's private store buffer, unless those two harts are the only two in the system.
Nor may a cache coherence protocol forward a value from one hart to another until the coherence protocol has invalidated all older copies from other caches.
Of course, microarchitectures may (and high-performance implementations likely will) violate these rules under the covers through speculation or other optimizations, as long as any non-compliant behaviors are not exposed to the programmer.

As a rough guideline for interpreting the PPO rules in RVWMO, we expect the following from the software perspective:
\begin{itemize}
  \item programmers will use PPO rules \ref{ppo:->st}--\ref{ppo:amoload} regularly and actively.
  \item expert programmers will use PPO rules \ref{ppo:addr}--\ref{ppo:ctrl} to speed up critical paths of important data structures.
  %\item expert programmers will occasionally use PPO rules \ref{ppo:rdw}--\ref{ppo:rfiaq} in very aggressive code and/or as part of a longer chain of synchronization.
  \item even expert programmers will rarely if ever use PPO rules \ref{ppo:rdw}--\ref{ppo:addrpo} directly.  These are included to facilitate common microarchitectural optimizations (rule~\ref{ppo:rdw}) and the operational formal modeling approach (rules \ref{ppo:ld->st->ld}--\ref{ppo:addrpo}) described in Section~\ref{sec:operational}.  They also facilitate the process of porting code from other architectures which have similar rules.
\end{itemize}

We also expect the following from the hardware perspective:
\begin{itemize}
  \item PPO rules \ref{ppo:->st}--\ref{ppo:release} and \ref{ppo:amostore}--\ref{ppo:amoload} reflect well-understood rules that should pose few surprises to architects.
  \item PPO rule \ref{ppo:strongacqrel} may not be immediately obvious to architects, but is somewhat standard nevertheless
  \item The load value axiom, the atomicity axiom, and PPO rules \ref{ppo:addr}--\ref{ppo:ctrl} and \ref{ppo:ld->st->ld}--\ref{ppo:addrpo} reflect rules that most hardware implementations will enforce naturally, unless they contain extreme optimizations.  Of course, implementations should make sure to double check these rules nevertheless.  Hardware must also ensure that syntactic dependencies are not ``optimized away''.
  %\item PPO rules \ref{ppo:strongacqrel} and \ref{ppo:rfiaq} may not be obvious or intuitive, and hence they deserve particular attention.
  %\item PPO rules \ref{ppo:strongacqrel} and \ref{ppo:rmwrfi}--\ref{ppo:rfiaq} may not be obvious or intuitive, and hence they deserve particular attention.
  %\item PPO rule \ref{ppo:success} is not obvious, but it is necessary to avoid certain out-of-thin-air-like behavior that appears with store-conditional success values
  \item PPO rule \ref{ppo:rdw} reflects a natural and common hardware optimization, but one that is very subtle and hence is worth double checking carefully.
\end{itemize}

Architectures are free to implement any of the memory model rules as conservatively as they choose.  For example, a hardware implementation may choose to do any or all of the following:
  \begin{itemize}
    \item interpret all fences as if they were {\tt fence rw,rw} (or {\tt fence iorw,iorw}, if I/O is involved), regardless of the bits actually set
    \item implement all fences with {\tt .pw} and {\tt .sr} as if they were {\tt fence~rw,rw} (or {\tt fence~iorw,iorw}, if I/O is involved), as ``{\tt w,r}'' is the most expensive of the four possible normal memory orderings anyway
    \item ignore any addresses passed to a fence instruction and simply implement the fence for all addresses
    \item implement an instruction with {\tt .aq} set as being preceded immediately by {\tt fence r,rw}
    \item implement an instruction with {\tt .rl} set as being succeeded immediately by {\tt fence rw,w}
    \item enforcing all same-address load-load ordering, even in the presence of patterns such as ``fri-rfi'' and ``RSW''
    \item forbid any forwarding of a value from a store in the store buffer to a subsequent AMO or {\tt lr} to the same address
    \item forbid any forwarding of a value from an AMO or {\tt sc} in the store buffer to a subsequent load to the same address
    \item implement TSO on all memory accesses, and ignore any normal memory fences that do not include ``{\tt w,r}'' ordering
    \item implement all atomics to be RCsc; i.e., always enforce all store-release-to-load-acquire ordering
  \end{itemize}
%PPO rules~\ref{ppo:ld->st->ld}--\ref{ppo:addrpocfence} are not intended to impose any ordering requirements onto a processor pipeline beyond constraints which arise naturally, but extremely-optimized pipelines should be careful not to violate these rules nevertheless (or to ensure that any speculation-based optimizations do not make illegal behaviors visible to software). 

Architectures which implement RVTSO can safely do the following:
\begin{itemize}
  \item Ignore all {\tt .aq} and {\tt .rl} bits, since these are implicitly always set under RVTSO.  ({\tt .aqrl} cannot be ignored, however, due to PPO rules \ref{ppo:strongacqrel}--\ref{ppo:amoload}.)
  \item Ignore all fences which do not have both {\tt .pw} and {\tt .sr} (unless the fence also orders I/O)
  \item Ignore PPO rules \ref{ppo:->st} and \ref{ppo:addr}--\ref{ppo:addrpo}, since these are redundant with other PPO rules under RVTSO assumptions
\end{itemize}

Other general notes:

\begin{itemize}
  \item Silent stores (i.e., stores which write the same value that already exists at a memory location) do not have any special behavior from a memory model point of view.  Microarchitectures that attempt to implement silent stores must take care to ensure that the memory model is still obeyed, particularly in cases such as RSW (Section~\ref{sec:ppo:rdw}) which tend to be incompatible with silent stores.
  \item Writes may be merged (i.e., two consecutive writes to the same address may be merged) or subsumed (i.e., the earlier of two back-to-back writes to the same address may be elided) as long as the resulting behavior does not otherwise violate the memory model semantics.
\end{itemize}

The question of write subsumption can be understood from the following example:
\begin{figure}[h!]
  \centering
  {
    \tt\small
    \begin{tabular}{cl||cl}
    \multicolumn{2}{c}{Hart 0} & \multicolumn{2}{c}{Hart 1} \\
    \hline
        & li t1, 3    &     & li  t3, 2    \\
        & li t2, 1    &     &              \\
    (a) & sw t1,0(s0) & (d) & lw  a0,0(s1) \\
    (b) & fence w, w  & (e) & sw  a0,0(s0) \\
    (c) & sw t2,0(s1) & (f) & lw  t3,0(s0) \\
    \end{tabular}
  }
  ~~~~
  \diagram
  \caption{Write subsumption litmus test}
  \label{fig:litmus:subsumption}
\end{figure}

As written, (a) must precede (f) in the global memory order:
\begin{itemize}
  \item (a) precedes (c) in the global memory order because of rule 2
  \item (c) precedes (d) in the global memory order because of the Load Value axiom
  \item (d) precedes (e) in the global memory order because of rule 7
  \item (e) precedes (f) in the global memory order because of rule 1
\end{itemize}

A very aggressive microarchitecture might erroneously decide to discard (e), as (f) supersedes it, and this may in turn lead the microarchitecture to break the now-eliminated dependency between (d) and (f) (and hence also between (a) and (f)).
This would violate the memory model rules, and hence it is forbidden.
Write subsumption may in other cases be legal, if for example there were no data dependency between (d) and (e).

\section{Summary of New/Modified ISA Features}

At a high level, PPO rules \ref{ppo:strongacqrel} and \ref{ppo:ld->st->ld}--\ref{ppo:addrpo} are new rules that did not exist in the original ISA spec.  Rule~\ref{ppo:rdw} and the specifics of the atomicity axiom were addressed but not stated in detail.

Other new or modified ISA details:
\begin{itemize}
  \item There is an RCpc ({\tt .aq} and {\tt .rl}) vs.\@ RCsc ({\tt .aqrl}) distinction
  \item Load-release and store-acquire are deprecated
  \item {\tt lr}/{\tt sc} behavior was clarified
  %\item Fences reserve two bits for platform-specific use
\end{itemize}

\subsection{Possible Future Extensions}

We expect that any or all of the following possible future extensions would be compatible with the RVWMO memory model:

\begin{itemize}
  \item `V' vector ISA extensions
  \item A transactional memory subset of the `T' ISA extension
  \item `J' JIT extension
  \item Native encodings for {\tt l\{b|h|w|d\}.aq}/{\tt s\{b|h|w|d\}.rl}
  \item Fences limited to certain addresses
  \item Cache writeback/flush/invalidate/etc.\@ hints, but these should be considered hints, not functional requirements.  Any cache management operations which are required for basic correctness should be described as (possibly address range-limited) fences to comply with the RISC-V philosophy (see also {\tt fence.i} and {\tt sfence.vma}).  For example, a functional cache writeback instruction might instead be written as ``{\tt fence~rw[addr],w[addr]}''.
\end{itemize}

\section{Litmus Tests}

These litmus tests represent some of the better-known litmus tests in the field, plus some tests that are randomly-generated, plus some tests that are generated to be particularly relevant to the RVWMO memory model.

All will be made available for download once they are generated.

We expect that these tests will one day serve as part of a compliance test suite, and we expect that many architects will use them for verification purposes as well.

COMING SOON!

\section{TO-DO}
\begin{itemize}
  \item Explain all the terms relating to dependencies
\end{itemize}

\chapter{Formal Memory Model Specifications}

\begin{commentary}
  To facilitate formal analysis of RVWMO, we present a set of formalizations in this chapter.  Any discrepancies are unintended; the expectation is that the models will describe exactly the same sets of legal behaviors, pending some memory model changes that have not quite been added to all of the formalizations yet.

  As such, these formalizations should be considered snapshots from some point in time during the development process rather than finalized specifications.

  At this point, no individual formalization is considered authoritative, but we may designate one as such in collaboration with the ISA specification and/or formalization task groups.
\end{commentary}

\section{Formal Axiomatic Specification in Alloy}
\label{sec:alloy}

We present a formal specification of the RVWMO memory model in Alloy (\url{http://alloy.mit.edu}).
This model is available online at \url{https://github.com/daniellustig/riscv-memory-model}.

\begin{figure}[h!]
  {
  \tt\bfseries\centering\footnotesize
  \begin{lstlisting}
////////////////////////////////////////////////////////////////////////////////
// =RISC-V RVWMO axioms=

// Preserved Program Order
fun ppo : Event->Event {
  // same-address ordering
  po_loc :> Store

  // explicit synchronization
  + ppo_fence
  + Load.aq <: ^po
  + ^po :> Store.rl
  + Store.aq.rl <: ^po :> Load.aq.rl
  + ^po :> Load.sc
  + Store.sc <: ^po

  // dependencies
  + addr
  + data
  + ctrl :> Store
  + (addr+data).successdep

  // RDW
  + (po_loc & (fre.rfe))

  // pipeline dependency artifacts
  + (addr+data).rfi
  + addr.^po :> Store
  + ctrl.(FenceI <: ^po)
  + addr.^po.(FenceI <: ^po)
}

// the global memory order respects preserved program order
fact { ppo in gmo }
\end{lstlisting}}
  \caption{The RVWMO memory model formalized in Alloy (1/4: PPO)}
  \label{fig:alloy1}
\end{figure}
\begin{figure}[h!]
  {
  \tt\bfseries\centering\footnotesize
  \begin{lstlisting}
// Load value axiom
fun candidates[r: Load] : set Store {
  (r.~gmo & Store & same_addr[r]) // writes preceding r in gmo
  + (r.^~po & Store & same_addr[r]) // writes preceding r in po
}

fun latest_among[s: set Event] : Event { s - s.~gmo }

pred LoadValue {
  all w: Store | all r: Load |
    w->r in rf <=> w = latest_among[candidates[r]]
}

fun after_reserve_of[r: Load] : Event { latest_among[r + r.~rf].gmo }

pred Atomicity {
  all r: Store.~rmw |               // starting from the read r of an atomic,
    no x: Store & same_addr[r + r.rmw] | // there is no write x to the same addr
      x not in same_hart[r]         // from a different hart, such that
      and x in after_reserve_of[r]  // x follows (the write r reads from) in gmo
      and r.rmw in x.gmo            // and r follows x in gmo
}

pred RISCV_mm { LoadValue and Atomicity }
\end{lstlisting}}
  \caption{The RVWMO memory model formalized in Alloy (2/4: Axioms)}
  \label{fig:alloy2}
\end{figure}
\begin{figure}[h!]
  {
  \tt\bfseries\centering\footnotesize
  \begin{lstlisting}
////////////////////////////////////////////////////////////////////////////////
// Basic model of memory

sig Hart {  // hardware thread
  start : one Event
}
sig Address {}
abstract sig Event {
  po: lone Event // program order
}

abstract sig MemoryEvent extends Event {
  address: one Address,
  aq: lone MemoryEvent, // opcode bit
  rl: lone MemoryEvent, // opcode bit
  sc: lone MemoryEvent, // for AMOs with .aq and .rl, to distinguish from lr/sc
  gmo: set MemoryEvent   // global memory order
}
sig Load extends MemoryEvent {
  addr: set Event,
  ctrl: set Event,
  data: set Store,
  successdep: set Event,
  rmw: lone Store
}
sig Store extends MemoryEvent {
  rf: set Load
}
sig Fence extends Event {
  pr: lone Fence, // opcode bit
  pw: lone Fence, // opcode bit
  sr: lone Fence, // opcode bit
  sw: lone Fence  // opcode bit
}
sig FenceI extends Event {}

// FENCE PPO
fun FencePRSR : Fence { Fence.(pr & sr) }
fun FencePRSW : Fence { Fence.(pr & sw) }
fun FencePWSR : Fence { Fence.(pw & sr) }
fun FencePWSW : Fence { Fence.(pw & sw) }

fun ppo_fence : MemoryEvent->MemoryEvent {
    (Load  <: ^po :> FencePRSR).(^po :> Load)
  + (Load  <: ^po :> FencePRSW).(^po :> Store)
  + (Store <: ^po :> FencePWSR).(^po :> Load)
  + (Store <: ^po :> FencePWSW).(^po :> Store)
}
\end{lstlisting}}
  \caption{The RVWMO memory model formalized in Alloy (3/4: model of memory)}
  \label{fig:alloy3}
\end{figure}
\begin{figure}[h!]
  {
  \tt\bfseries\centering\footnotesize
  \begin{lstlisting}
// auxiliary definitions
fun RFInit : Load { Load - Store.rf }

fun po_loc : Event->Event { ^po & address.~address }
fun same_hart[e: Event] : set Event { e + e.^~po + e.^po }
fun same_addr[e: Event] : set Event { e.address.~address }

// basic facts about well-formed execution candidates
fact { acyclic[po] }
fact { all e: Event | one e.*~po.~start }  // each event is in exactly one hart
fact { rf.~rf in iden } // each read returns the value of only one write
fact { total[gmo, MemoryEvent] } // gmo is a total order over all MemoryEvents

//rf
fact { rf in address.~address }
fun internal : Event->Event { Event <: (*po + *~po) :> Event }
fun rfi : Store->Load { rf & internal }
fun rfe : Store->Load { rf - internal }
fun fr : Load->Store {
  ((~rf.gmo) + (RFInit <: Event->Event)) & (address.~address :> Store)
}
fun fre : Load->Store { fr - internal }

//dep
fact { addr + ctrl + data in ^po }
fact { successdep in (Store.~rmw) <: ^po }
fact { ctrl.*po in ctrl }
fact { rmw in ^po }

// to unclutter the display a bit
fun mo : MemoryEvent->MemoryEvent {
  gmo - (gmo.gmo)
}

////////////////////////////////////////////////////////////////////////////////
// =Opcode encoding restrictions=

// opcode bits are either set (encoded, e.g., as f.pr in iden) or unset
// (f.pr not in iden).  The bits cannot be used for anything else
fact { pr + pw + sr + sw + aq + rl + sc in iden }
fact { sc in aq & rl }
fact { Load.sc.rmw in Store.sc and Store.sc.~rmw in Load.sc }

// Fences must have either pr or pw set, and either sr or sw set
fact { Fence in Fence.(pr + pw) & Fence.(sr + sw) }

// there is no write-acquire, but there is write-strong-acquire
fact { Store.aq in Store.aq.rl }
fact { Load.rl in Load.aq.rl }

////////////////////////////////////////////////////////////////////////////////
// =Alloy shortcuts=
pred acyclic[rel: Event->Event] { no iden & ^rel }
pred total[rel: Event->Event, bag: Event] {
  all disj e, e': bag | e->e' in rel + ~rel
  acyclic[rel]
}
\end{lstlisting}}
  \caption{The RVWMO memory model formalized in Alloy (4/4: Auxiliaries)}
  \label{fig:alloy4}
\end{figure}


\clearpage
\section{Formal Axiomatic Specification in Herd}

See also \url{http://moscova.inria.fr/~maranget/cats7/riscv} for a lot of additional supplementary information.

\begin{figure}[h!]
  {
  \tt\bfseries\centering\footnotesize
  \begin{lstlisting}
(*************)
(* Utilities *)
(*************)

let fence.r.r = [R];fencerel(Fence.r.r);[R]
let fence.r.w = [R];fencerel(Fence.r.w);[W]
let fence.r.rw = [R];fencerel(Fence.r.rw);[M]
let fence.w.r = [W];fencerel(Fence.w.r);[R]
let fence.w.w = [W];fencerel(Fence.w.w);[W]
let fence.w.rw = [W];fencerel(Fence.w.rw);[M]
let fence.rw.r = [M];fencerel(Fence.rw.r);[R]
let fence.rw.w = [M];fencerel(Fence.rw.w);[W]
let fence.rw.rw = [M];fencerel(Fence.rw.rw);[M]

let fence = 
  fence.r.r | fence.r.w | fence.r.rw |
  fence.w.r | fence.w.w | fence.w.rw |
  fence.rw.r | fence.rw.w | fence.rw.rw


let po-loc-no-w = po-loc \ (po-loc;[W];po-loc)
let rsw = rf^-1;rf

let LD-ACQ = R & (Acq|AcqRel)
and ST-REL = W & (Rel|AcqRel)
and LD-ACQ-SC = R & AcqRel
and ST-REL-SC = W & AcqRel

let ST-SC = W & Sc
and LD-SC = R & Sc

(*************)
(* ppo rules *)
(*************)

let r1 = [M];po-loc;[W]
and r2 = fence
and r3 = [LD-ACQ];po;[M]
and r4 = [M];po;[ST-REL]
and r5 = [ST-REL-SC];po;[LD-ACQ-SC]
and r6 = [ST-SC];po;[M]
and r7 = [M];po;[LD-SC]
and r8 = [R];addr;[M]
and r9 = [R];data;[W]
and r10 = [R];ctrl;[W]
and r11 = depend;instr;success
and r12 = ([R];po-loc-no-w;[R]) \ rsw
and r13 = [R];(addr|data);[W];po-loc-no-w;[R]
and r14 = [R];addr;[M];po;[W]
and r15 = [R];ctrl;[Fence.i];po;[R]
and r16 = [R];addr;[M];po;[Fence.i];po;[M]

let ppo =
 r1 | r2 | r3 | r4 | r5 | r6 | r7 | r8 | r9
| r10 | r11 | r12 | r13 | r14 | r15 | r16
\end{lstlisting}
  }
  \caption{{\tt riscv-defs.cat}, part of a herd version of the RVWMO memory model (1/3)}
  \label{fig:herd1}
\end{figure}

\begin{figure}[h!]
  {
  \tt\bfseries\centering\footnotesize
  \begin{lstlisting}
Total

(* Notice that herd has defined its own rf relation *)

(* Define ppo *)
include "riscv-defs.cat"

(********************************)
(* Generate global memory order *)
(********************************)

let gmo0 = (* precursor: ie build gmo as an total order that include gmo0 *)
  loc & (W\FW) * FW | # Final write before any write to the same location
  ppo |               # ppo compatible
  rfe                 # first half of 

(* Walk over all linear extensions of gmo0 *)
with  gmo from linearisations(M\IW,gmo0)

(* Add initial writes upfront -- convenient for computing rfGMO *)
let gmo = gmo | loc & IW * (M\IW)

(**********)
(* Axioms *)
(**********)

(* Compute rf according to the load value axiom, aka rfGMO *)
let WR = loc & ([W];(gmo|po);[R])
let rfGMO = WR \ (loc&([W];gmo);WR)

(* Check equality of herd rf and of rfGMO *)
empty (rf\rfGMO)|(rfGMO\rf) as RfCons

(* Atomic axion *)
let infloc = (gmo & loc)^-1
let inflocext = infloc & ext

let winside  = (infloc;rmw;inflocext) & (infloc;rf;rmw;inflocext) & [W]
empty winside as Atomic
\end{lstlisting}
  }
  \caption{{\tt riscv.cat}, a herd version of the RVWMO memory model (2/3)}
  \label{fig:herd2}
\end{figure}

\begin{figure}[h!]
  {
  \tt\bfseries\centering\footnotesize
  \begin{lstlisting}
Partial

(***************)
(* Definitions *)
(***************)

(* Define ppo *)
include "riscv-defs.cat"

(* Compute coherence relation *)
include "cos-opt.cat"

(**********)
(* Axioms *)
(**********)

(* Sc per location *)
acyclic co|rf|fr|po-loc as Coherence

(* Main model axiom *)
acyclic co|rfe|fr|ppo as Model

(* Atomicity axiom *)
empty rmw & (fre;coe) as Atomic
\end{lstlisting}
  }
  \caption{{\tt riscv.cat}, part of an alternative herd presentation of the RVWMO memory model (3/3)}
  \label{fig:herd3}
\end{figure}

%% Operational Memory Model
\clearpage
\section{An Operational Memory Model}
\label{sec:operational}
This is an alternative presentation of the RVWMO memory model in
operational style.
%
It aims to admit exactly the same extensional behaviour as the
axiomatic presentation: for any given program, admitting an execution
if and only if the axiomatic presentation allows it.

The axiomatic presentation is defined as a predicate on complete
candidate executions.  In contrast, this operational presentation has
an abstract microarchitectural flavour: it is expressed as a state
machine, with states that are an abstract representation of hardware
machine states, and with explicit out-of-order and speculative
execution
(but abstracting from more implementation-specific microarchitectural
details such as register renaming, store buffers, cache hierarchies, cache protocols, etc.).
As such, it can provide useful intuition.
It can also
construct executions incrementally, making it possible to
interactively and randomly explore the behaviour of larger examples,
while the axiomatic model requires complete candidate executions
over which the axioms can be checked.

The operational presentation covers mixed-size execution, with
potentially overlapping memory accesses of different power-of-two byte
sizes.  Misaligned accesses are broken up into single-byte accesses.

An interactive version of the model, together with a library of litmus tests,
is provided online: \url{http://www.cl.cam.ac.uk/~pes20/rmem}.
This is integrated with a fragment of the RISC-V ISA semantics
(RV64I and A) expressed explicitly in Sail
(\url{https://github.com/rems-project/sail})).
% TODO: compare with the herd and alloy versions


Below is an informal introduction of the model states and transitions.
The description of the formal model starts in the next subsection.


Terminology: The terms ``load-acquire'' and ``store-release'' are used
to describe  memory operations (as in
Section~\ref{sec:memprimitives}) and to describe instructions.
In particular, ``load-acquire'' includes AMO instructions with {\tt .aq} set, and ``store-release'' includes all {\em sc.w.rl/sc.d.rl} instructions, even when the store operation fails.
In a similar way, when the terms ``load'' and ``store'' are used in the context of instructions, they refer to instructions that may generate those operations (i.e.~they also cover AMOs).



\paragraph{Model states}
A model state consists of a shared memory and a tuple of hart states.
\begin{center}
\sffamily
\begin{tabular}{ccc}
\cline{1-1}\cline{3-3}
\multicolumn{1}{|c|}{Hart 0} & \bf \dots & \multicolumn{1}{|c|}{Hart $n$} \\
\cline{1-1}\cline{3-3}
$\big\uparrow$ $\big\downarrow$ & & $\big\uparrow$ $\big\downarrow$ \\
\hline
\multicolumn{3}{|c|}{Shared Memory} \\
\hline
\end{tabular}
\end{center}
The shared memory state records the most recent memory store operation to each location.
To handle atomic memory accesses ({\em  lr}, {\em sc} and AMOs), the
memory is extended with a map (the \emph{atomics map}) from memory
load operations to sets of store slices (memory store operations with subsets of
their byte indices), associating a load operation of an atomic load with the store slices it reads from (excluding stores that have been forwarded to the load and have not reached memory yet).

Each hart state consists principally of a tree of instruction instances, some of which have been \emph{finished}, and some of which have not.
Non-finished instruction instances can be subject to \emph{restart}, e.g.~if they depend on an out-of-order or speculative load that turns out to be unsound.
AMOs are treated slightly different in that respect.
The load part of an AMO can be marked as finished before the entire instruction is finished; when such instruction is restarted it is rolled back to the state it was in when the load part was marked as finished.

Conditional branch and indirect jump instructions may have multiple successors in the instruction tree.
When such instruction is finished, any un-taken alternative paths are discarded.

Each instruction instance in the instruction tree has a state that
includes an execution state of the intra-instruction semantics (the
ISA pseudocode for this instruction).
The model uses a formalisation of the intra-instruction semantics in Sail.
One can think of the execution state of an instruction as a representation of the pseudocode control state, pseudocode call stack, and local variable values.
An instruction instance state also includes information about the instance's memory and register footprints, its register reads and writes, its memory operations, whether it is finished, etc.

\paragraph{Model transitions}
The model defines, for any model state, the set of allowed transitions, each of which is a single atomic step to a new abstract machine state.
Execution of a single instruction will typically involve many
transitions, and they may be interleaved in operational-model
execution with transitions arising from other instructions. 
Each transition arises from a single instruction instance; it will
change the state of that instance, and it may depend on or change the
rest of its hart state and the shared memory state, but it does not depend on other hart states, and it will not change them.
% Instructions cannot be treated
% as atomic units: complete execution of a single instruction instance may
% involve many transitions, which can be interleaved with those of other
% instances in the same or other harts, and some of this is programmer-visible.
The transitions are introduced below and defined in Section~\ref{sec:omm:transitions}, with a precondition and a construction of the post-transition model state for each.

\noindent Transitions for all instructions:
\begin{itemize}
\item \nameref{omm:fetch}: This transition represents a fetch and
  decode of a new instruction instance, as a program order successor
  of a previously fetched instruction instance (or the initial fetch
  address).

The model assumes the instruction memory is fixed; it does not
describe the behaviour of self-modifying code. 
In particular, the \nameref{omm:fetch} transition does not generate memory load operations, and the shared memory is not involved in the transition.
Instead, the model depends on an external oracle that provides an opcode when given a memory location.
%


%\fixme{why is Fetch not eager? is it because of loops? is that a good reason? I think fetch should be eager and just add a comment to the description of eager below saying fetching needs some attention}
\item[$\circ$] \nameref{omm:reg_write}: This is a write of a register value.
\item[$\circ$] \nameref{omm:reg_read}: This is a read of a register
  value from the most recent program-order-predecessor instruction instance that writes to that register.
\item[$\circ$] \nameref{omm:sail_interp}: This covers pseudocode
  internal computation: arithmetic, function calls, etc.
\item[$\circ$] \nameref{omm:finish}: At this point the instruction pseudocode is done, the instruction cannot be restarted or discarded, and all memory effects have taken place.
For conditional branch and indirect jump instructions, any program order successors that were fetched from an address that is not the one that was written to the {\em pc} register are discarded, together with the sub-tree of instruction instances below them.
\end{itemize}

\noindent Transitions specific to load instructions:
\begin{itemize}
\item[$\circ$] \nameref{omm:initiate_load}: At this point the memory
  footprint of the load instruction is provisionally known (it could change if
  earlier instructions are restarted) and its individual memory load operations can start being satisfied.
\item \nameref{omm:sat_by_forwarding}: This partially or entirely
  satisfies a single memory load operation by forwarding, from
  program-order-previous memory store operations.
\item \nameref{omm:sat_from_mem}: This entirely satisfies the outstanding slices of a single memory load operation, from memory.
\item[$\circ$] \nameref{omm:complete_loads}: At this point all the memory load operations of the instruction have been entirely satisfied and the instruction pseudocode can continue executing.
A load instruction can be subject to being restarted until the \nameref{omm:finish} transition or \nameref{omm:finish_load_part}.
But, under some conditions, the model might treat a load instruction as non-restartable (e.g.~see \nameref{omm:prop_store}).
\end{itemize}



\noindent Transitions specific to store instructions:
\begin{itemize}
\item[$\circ$] \nameref{omm:announce_store_footprint}: At this point the memory footprint of the store is provisionally known.
\item[$\circ$] \nameref{omm:initiate_store}: At this point the memory store operations have their values and program-order-subsequent memory load operations can be satisfied by forwarding from them.
\item[$\circ$] \nameref{omm:commit_stores}: At this point the store operations are guaranteed to happen (the instruction can no longer be restarted or discarded), and they can start being propagated to memory.
\item \nameref{omm:prop_store}: This propagates a single memory store operation to memory.
\item[$\circ$] \nameref{omm:complete_stores}: At this point all the memory store operations of the instruction have been propagated to memory, and the instruction pseudocode can continue executing.
\end{itemize}

\noindent Transitions specific to {\em  sc} instructions:
\fixme{these are not needed if we decide the store operation is ordered with the register write}
\begin{itemize}
\item \nameref{omm:excl_success}: This commits to the success of the {\em sc}.
\item \nameref{omm:excl_fail}: This commits to the failure of the {\em sc}.
\end{itemize}

\noindent Transitions specific to AMO instructions:
\begin{itemize}
\item[$\circ$] \nameref{omm:finish_load_part}: At this point the load part of an AMO instruction is done, and the load part cannot be restarted or discarded. If the AMO instruction is restarted after the transition is taken, the instruction rolls back to its state right after this transition.
\end{itemize}

\noindent Transitions specific to fence instructions:
\begin{itemize}
\item[$\circ$] \nameref{omm:commit_barrier}
\end{itemize}

%\begin{commentary}
The transitions labelled~$\circ$ can always be taken eagerly, as soon as their precondition is satisfied, without excluding other behaviour; the $\bullet$ cannot.
%\end{commentary}

%\begin{discussion}
An instance of a load instruction, after being fetched, will typically experience the following transitions in this order:
\begin{enumerate}
\item \nameref{omm:reg_read}
\item \nameref{omm:initiate_load}
\item \nameref{omm:sat_by_forwarding} and/or \nameref{omm:sat_from_mem} (as many as needed to satisfy all the load operations of the instance)
\item \nameref{omm:complete_loads}
\item \nameref{omm:reg_write}
\item \nameref{omm:finish}
\end{enumerate}
Before, between and after the transitions above, any number of \nameref{omm:sail_interp} transitions may appear.
In addition, a \nameref{omm:fetch} transition for fetching the instruction in the next program location will be available until it is taken.
%\end{discussion}



This concludes the informal description of the operational model.
The following sections describe the formal operational model.

\subsection{Intra-instruction Pseudocode Execution}
The intra-instruction semantics for each instruction instance is expressed as a state machine, essentially running the instruction pseudocode.
Given a pseudocode execution state, it computes the next state.  Most
states identify a pending memory or register operation, requested by
the pseudocode, which the memory model has to do.  The
states are:

\begin{center}
\begin{tabular}{l@{ \quad-\quad }l}
{\sc Load\_mem}($kind$, $address$, $size$, $load\_continuation$)
    & memory load operation\\
{\sc Atomic\_res}($res\_continuation$)
    & {\em sc} result \fixme{remove?}\\
{\sc Store\_ea}($kind$, $address$, $size$, $next\_state$)
    & memory store effective address\\
{\sc Store\_memv}($memory\_value$, $store\_continuation$)
    & memory store value\\
{\sc Fence}($kind$, $next\_state$)
    & fence\\
{\sc Read\_reg}($reg\_name$, $read\_continuation$)
    & register read\\
{\sc Write\_reg}($reg\_name$, $register\_value$, $next\_state$)
    & register write\\
{\sc Internal}($next\_state$)
    & pseudocode internal step\\
{\sc Done}
    & end of pseudocode\\
\end{tabular}
\end{center}
Here:
\begin{tightlist}
\item $memory\_value$ and $register\_value$ are lists of bytes;
\item $address$ is an integer of XLEN bits;
\item for load/store, $kind$ identifies whether it is regular or atomic ({\em lr/sc} and AMOs), and the type of load/store (acquire/release-RCpc, acquire/release-RCsc, SC, plain);
\item for fence, $kind$ identifies the predecessor and successor sets of the instruction;
\item $reg\_name$ identifies a register and a slice thereof (start and
  end bit indices); and
  \item the continuations describe how the instruction instance will continue for each value that might be provided by the surrounding memory model (the $load\_continuation$ and $read\_continuation$ take the value loaded from memory and read from the previous register write, and $store\_continuation$ takes $false$ for a {\em sc} that failed and $true$ in all other cases).
\end{tightlist}
    
\begin{commentary}
For example, given the load instruction \verb!lw x1,0(x2)!,
an execution will typically go as follows.
The initial execution state will be computed from the pseudocode for the given opcode.
This can be expected to be {\sc Read\_reg}({\tt x2}, $read\_continuation$).
Feeding the most recently written value of register {\tt x2}, say 0x4000, to $read\_continuation$ returns
{\sc Load\_mem}({\tt plain\_load}, 0x4000, 4, $load\_continuation$)
(the instruction semantics will be blocked
if necessary until the register value is available).
Feeding the 4-byte value loaded from memory location 0x4000, say 0x42, to $load\_continuation$ returns
{\sc Write\_reg}({\tt x1}, 0x42, {\sc Done}).
Many {\sc Internal}($next\_state$) states may appear before and between the states above.
\end{commentary}

Notice that stores are split into two steps, {\sc Store\_ea} and {\sc Store\_memv}: the first one makes the memory footprint of the store provisionally known, and the second one adds the value to be stored.
We ensure these are paired in the pseudocode ({\sc Store\_ea} followed by {\sc Store\_memv}), but there may be other steps between them.
\begin{commentary}
It is observable that the {\sc Store\_ea} can occur before the value to be stored is determined.
For example, for the litmus test LB+fence.r.rw+data-po to be allowed by the operational model (as it is by RVWMO), the first store in Hart 1 has to take the {\sc Store\_ea} step before its value is determined, so that the second store can see it is to a non-intersecting memory footprint, allowing the second store to be committed out of order without violating coherence.
\end{commentary}

The pseudocode of each instruction performs at most one store or one load, except for AMOs that perform exactly one load and one store.
Those memory accesses are then split apart into the architecturally atomic units by the hart semantics (see \nameref{omm:initiate_load} and \nameref{omm:announce_store_footprint} below).

Informally, each bit of a register read should be satisfied from a register write by the most recent (in program order) instruction instance that can write that bit (or from the hart's initial register state if there is no such write).
Hence, it is essential to know the register write footprint of each instruction instance, which we calculate when the instruction instance is created (see the action of \nameref{omm:fetch} below).
We ensure in the pseudocode that each instruction does at most one register write to each register bit, and also that it does not try to read a register value it just wrote.

Data-flow dependencies (address and data) in the model emerge from the
fact that each register read has to wait for the appropriate register write to be executed (as described above).

\subsection{Instruction Instance State}\label{sec:omm:inst_state}
Each instruction instance $i$ has a state comprising:
\begin{itemize}
\item $program\_loc$, the memory address from which the instruction was fetched;
\item $instruction\_kind$, identifying whether this is a load, store, AMO, or fence instruction, each with the associated kind; or a branch; or a `simple' instruction;
\item $src\_regs$, the set of source $reg\_name$s (including system registers), as statically determined from the pseudocode of the instruction;
\item $dst\_regs$, the destination $reg\_name$s (including system registers), as statically determined from the pseudocode of the instruction;
\item $pseudocode\_state$ (or sometimes just `state' for short), one of:
  \begin{center}
  \begin{tabular}{l@{ \quad-\quad }l}
  {\sc Plain} $next\_state$                        & ready to make a pseudocode transition \\
  {\sc Pending\_mem\_loads} $load\_continuation$   & requesting memory load operation(s) \\
  {\sc Pending\_mem\_stores} $store\_continuation$ & requesting memory store operation(s) \\
%   {\sc Pending\_exception} $exception$             & performing an exception;
  \end{tabular}
  \end{center}
\fixme{add parens, and get rid of quotes below}

\item $reg\_reads$, the register reads the instance has performed, including, for each one, the register write slices it read from;
\item $reg\_writes$, the register writes the instance has performed;
\item $mem\_loads$, a set of memory load operations, and for each one
  the as-yet-unsatisfied slices (the byte indices that have not been
  satisfied yet), and, for the satisfied slices, the store slices
  (each consisting of a memory store operation and subset of its byte indices) that satisfied it.
\item $mem\_stores$, a set of memory store operations, and for each one a flag that indicates whether it has been propagated (passed to the shared memory) or not.
\item $successful\_atomic$, for {\em sc}, indicates whether the instruction is committed to succeed or fail, or no commitment has been made yet; for AMOs this will always be set to true (committed to succeed).
\item information recording whether the instance is committed, finished, etc.
\end{itemize}

Each memory load operation includes a load kind and a memory footprint (address and size).
Each memory store operations includes a store kind, a memory footprint, and, when available, a value.

A load instruction instance with a non-empty $mem\_loads$, for which all the load operations are satisfied (i.e.~there are no unsatisfied load slices) is said to be {\it entirely satisfied}.

Informally, an instruction instance is said to have {\it fully determined data} if the load instructions feeding its source registers are finished.
Similarly, it is said to have a {\it fully determined memory footprint} if the load instructions feeding its memory operation address register are finished.
%
Formally, we first define the notion of {\it fully determined register write}: a register write $w$ from $reg\_writes$ of instruction instance $i$ is said to be {\it fully determined} if one of the following conditions hold:
\begin{enumerate}
\item $i$ is finished (for AMOs just the load part\fixme{remove?}); or
\item the value written by $w$ is not affected by any memory operation that $i$ has made, and, for every register read that $i$ has made, that affects $w$, the register write from which $i$ read is fully determined.
\end{enumerate}
Now, an instruction instance $i$ is said to have  {\it fully determined data} if for every register read $r$ from $reg\_reads$, the register writes that $r$ reads from are fully determined.
An instruction instance $i$ is said to have a {\it fully determined memory footprint} if for every register read $r$ from $reg\_reads$ that feeds into $i$'s memory operation address, the register writes that $r$ reads from are fully determined.
\begin{commentary}
The {\tt rmem} tool records, for every register write, the set of register writes from other instructions that have been read by this instruction at the point of performing the write.
By carefully arranging the pseudocode of the instructions covered by the tool we were able to make it so that this is exactly the set of register writes on which the write depends on.
\end{commentary}

An {\em lr} instruction instance is called {\em successful} after the {\em sc} it is paired with (if such {\em sc} exists) is committed to succeed.
If a successful {\em lr} has a memory load operation that is mapped, in the atomics map, to a memory store operation slice $ws$, we say the {\em lr} has an outstanding lock on $ws$. \fixme{also the load part of AMO?}


\subsection{Hart State}
The model state of a single hart comprises:
\begin{itemize}
\item $hart\_id$, a unique identifier of the hart;
%\item $register\_data$, the name, bit width, and start bit index for each register;
\item $initial\_register\_state$, the initial register value for each register;
\item $initial\_fetch\_address$, the initial fetch address;
\item $instruction\_tree$, a tree of the instruction instances that have been fetched (and not discarded), in program order.
\end{itemize}


\subsection{Shared Memory State}
The model state of the shared memory comprises:
\begin{itemize}
\item $recent\_stores$, a map from memory locations to memory store operation slices:
each memory location is mapped to a one-byte slice of the most recent memory store operation  to that location.
\item $atomics$, a map from memory load operations to sets of memory
  store operation slices: 
the memory load operations of a successful atomic load are mapped to the store slices they read from, excluding slices that have been forwarded to the load from stores that have not been propagated yet.
\end{itemize}


\subsection{Transitions}\label{sec:omm:transitions}

Each of the paragraphs below describes a single kind of system transition.
The description starts with a condition over the current system state.
The transition can be taken in the current state only if the condition is satisfied.
The condition is followed by an action that is applied to that state when the transition is taken, in order to generate the new system state.

\paragraph{Fetch instruction}\label{omm:fetch}
A possible program-order-successor of instruction instance $i$ can be fetched from address $loc$ if:
\begin{enumerate}
\item it has not already been fetched, i.e., none of the immediate successors of $i$ in the hart's $instruction\_tree$ are from $loc$; and
\item $loc$ is a possible next fetch address for $i$:
  \begin{enumerate}
  \item for a conditional branch, the successor address and the branch target address;
  \item for an indirect jump instruction ({\em jalr}), when the target address is not yet determined, any address; and
  \item for any other instruction, the value written to the program counter register ({\em pc});
  \end{enumerate}
\end{enumerate}
\fixme{Does an instruction at the end of memory need special-case treatment?}

\fixme{Should we add explicit PC writes to the Sail ISA model for all
  instructions, or modify this text?}

Action: construct a freshly initialized instruction instance $i'$ for the instruction in the program memory at $loc$, with state ``{\sc Plain} $next\_state$'', computed from the instruction pseudocode, including the static information available from the pseudocode such as its $instruction\_kind$, $src\_regs$, and $dst\_regs$, and add $i'$ to the hart's $instruction\_tree$ as a successor of $i$.
% If the instruction fails to decode, set the state of $i'$ to ``{\sc Pending\_exception} $exception$'' with $exception$ that describes the decoding error.

\begin{commentary}
The possible next fetch addresses are available immediately after fetching $i$ and the model does not need to wait for the pseudocode to write to {\em pc} (as is the case with GPRs); this allows out-of-order execution, and speculation past conditional branches and jumps.
For most instructions these addresses are easily obtained from the instruction pseudocode.
The only exception to that is the indirect jump instruction ({\em jalr}), where the address depends on the value held in a register.
%
In principle the mathematical model should allow speculation to
arbitrary addresses here. 
%
The exhaustive search in the {\tt rmem} tool handles this by running the exhaustive search multiple times with a growing set of possible next fetch addresses for each indirect jump.
The initial search uses empty sets, hence there is no fetch after indirect jump instruction until the pseudocode of the instruction writes to {\em pc}, and then we use that value for fetching the next instruction.
Before starting the next iteration of exhaustive search, we collect for each indirect jump (grouped by code location) the set of values it wrote to {\em pc} in all the executions in the previous search iteration, and use that as possible next fetch addresses of the instruction.
This process terminates when no new fetch addresses are detected.
\end{commentary}

\paragraph{Initiate memory load operations}\label{omm:initiate_load}
An instruction instance $i$ with next pseudocode state {\sc Load\_mem}($kind$, $address$, $size$, $load\_continuation$) can initiate the corresponding memory load operations if:
\begin{enumerate}
\item all program-order-previous {\em fence} instructions with {\em .sr} set are finished;
\item all program-order-previous {\em fence.i} instructions are finished; \fixme{it was decided that fence.i does not have any memory model effects; remove this}
\item if $i$ is a load-acquire-RCsc, all program-order-previous store-releases-RCsc are finished; and
\item all non-finished program-order-previous load-acquire instructions are entirely satisfied.
\end{enumerate}
Action:
\begin{enumerate}
\item Construct the appropriate memory load operations $mlos$:
  \begin{itemize}
  \item if $address$ is aligned to $size$ then $mlos$ is a single memory load operation of $size$ bytes from $address$;
  \item otherwise, $mlos$ is a set of $size$ memory load operations, each of one byte, from the addresses $address\ldots address+size-1$.
  \end{itemize}
\item set $i.mem\_loads$ to $mlos$; and
\item update the state of $i$ to ``{\sc Pending\_mem\_loads} $load\_continuation$''.
\end{enumerate}

\begin{commentary}
In Section~\ref{sec:memprimitives} it is said that misaligned memory accesses may be decomposed at any granularity.
Here we decompose them to one-byte accesses as this granularity subsumes all others.
\end{commentary}

\paragraph{Satisfy memory load operation by forwarding from stores}\label{omm:sat_by_forwarding}
For a load instruction instance $i$ in state ``{\sc Pending\_mem\_loads} $load\_continuation$'', and a memory load operation, $mlo$ in $i.mem\_loads$ that has unsatisfied slices, the memory load operation can be partially or entirely satisfied by forwarding from unpropagated memory store operations by store instruction instances that are program-order-before $i$.

Let $msoss$ be the set of all unpropagated memory store operation slices from store instruction instances that are program-order-before $i$ (if $i$ is a load-acquire, exclude {\em sc} and AMO memory store operations \fixme{?}), that overlap with the unsatisfied slices of $mlo$, and which are not superseded by intervening stores that are either propagated or read from by this hart.
The last condition requires, for each memory store operation slice $msos$ in $msoss$ from instruction $i'$:
\begin{itemize}
\item that there is no store instruction program-order-between $i$ and $i'$ with a memory store operation overlapping $msos$; and
\item that there is no load instruction program-order-between $i$ and $i'$ that was satisfied from an overlapping memory store operation slice from a different hart.
\end{itemize}
Action:
\begin{enumerate}
\item update $mlo$ to indicate that it was satisfied by $msoss$; and
\item restart any speculative instructions which have violated coherence as a result of this, i.e., for every non-finished instruction $i'$ that is a program-order-successor of $i$, and every memory load operation $mlo'$ of $i'$ that was satisfied from $msoss'$, if there exists a memory store operation slice $msos'$ in $msoss'$, and an overlapping memory store operation slice from a different memory store operation in $msoss$, and $msos'$ is not from an instruction that is a program-order-successor of $i$, restart $i'$ and its data-flow dependents (including program-order-successors of restarted load-acquire instructions).
\end{enumerate}

\begin{commentary}
Forwarding memory store operations to a memory load might satisfy only some slices of the load, leaving other slices unsatisfied.

A consequence of the transition condition above is that store-release-RCsc memory store operations cannot be forwarded to load-acquires-RCsc:
a load-acquire-RCsc instruction cannot be in state ``{\sc Pending\_mem\_loads} $load\_continuation$'' before all the program-order-previous store-release-RCsc instructions are finished, and $msoss$ does not include memory store operations from finished stores (as those must be propagated memory store operations).
\end{commentary}


\paragraph{Satisfy memory load operation from memory}\label{omm:sat_from_mem}
For a load instruction instance $i$ in state ``{\sc Pending\_mem\_loads} $load\_continuation$'', and a memory load operation $mlo$ in $i.mem\_loads$, that has unsatisfied slices, the memory load operation can be satisfied from memory under the condition that if $i$ is an AMO or a successful {\em lr} then no other AMO or successful {\em lr} from a different hart has an outstanding lock on the memory store operations $mlo$ is trying to read from.
Action: let $msoss$ be the memory store operation slices from memory covering the unsatisfied slices of $mlo$, and apply the action of \nameref{omm:sat_by_forwarding}.
In addition, if $i$ is an AMO or a successful {\em lr}, update the atomics map by mapping $mlo$ to the union of $msoss$ and the set of memory store operation slices $mlo$ is already mapped to.

\begin{commentary}
Note that \nameref{omm:sat_by_forwarding} might leave some slices of the memory load operation unsatisfied.
\nameref{omm:sat_from_mem}, on the other hand, will always satisfy all the unsatisfied slices of the memory load operation.
\end{commentary}


\paragraph{Complete load operations}\label{omm:complete_loads}
A load instruction instance $i$ in state ``{\sc Pending\_mem\_loads} $load\_continuation$'' can be completed (not to be confused with finished) if all the memory load operations $i.mem\_loads$ are entirely satisfied (i.e.~there are no unsatisfied slices).
Action: update the state of $i$ to ``{\sc Plain} $next\_state$'', where $next\_state$ is the result of applying $load\_continuation$ to $memory\_value$, and $memory\_value$ is assembled from all the memory store operation slices that satisfied $i.mem\_loads$.


\paragraph{Finish load part of AMO instruction}\label{omm:finish_load_part}
An AMO instruction instance $i$ that has completed the load part, i.e., \nameref{omm:complete_loads} has been taken, can be marked as such if the conditions of \nameref{omm:finish} are satisfied, except for the fully determined data condition for register {\em rs2}.
Action: mark the load part of $i$ as finished. If later on $i$ has to be restarted, reset its state to the current state.


% \paragraph{Guarantee the success of {\em sc}}\label{omm:excl_success}
% A {\em sc} instruction instance $i$ with next pseudocode state {\sc Atomic\_res}($res\_cont$) can be guaranteed to succeed if:
% \begin{enumerate}
% \item $i$ has not been made to fail (as recorded in $i.successful\_atomic$);
% \item $i$ is paired with a {\em lr} $i'$; and
% \item if $i'$ has already been satisfied (not necessarily entirely), let $msoss$ be the set of propagated write slices $i'$ has read from, then, no slice in $msoss$ has been overwritten (in memory) by a write from a different hart, and no other AMO or successful {\em lr} from a different hart has an outstanding lock on a write slice from $msoss$.
% \end{enumerate}
% Action:
% \begin{enumerate}
% \item record in $i.successful\_atomic$ that $i$ will be successful;
% \item if $i'$ has already been satisfied, union $msoss$ with the set of write slices the memory load operation of $i'$ is mapped to in the atomics map, where $msoss$ is as above; and
% \item update the state of $i$ to ``{\sc Plain} $res\_cont(true)$''.
% \end{enumerate}
%
%
% \paragraph{Make a {\em sc} fail}\label{omm:excl_fail}
% A {\em sc} instruction instance $i$ with next pseudocode state {\sc Atomic\_res}($res\_cont$) can be made to fail if the {\em sc} has not been guaranteed to succeed (as recorded in $i.successful\_atomic$).
% Action:
% \begin{enumerate}
% \item record in $i.successful\_atomic$ that the {\em sc} was made to fail; and
% \item update the state of $i$ to ``{\sc Plain} $res\_cont(false)$''.
% \end{enumerate}
%
% \begin{commentary}
% Note that the promise-success transition is enabled before the {\em sc} commits, and we do not require it to have a fully-determined address or to be non-restartable.
% As a result, a {\em sc} that has already promised its success might be restarted.
% Since other instructions may rely on its promise, the restart will not affect the value of $i.successful\_atomic$.
% Instead, when the {\em sc} is restarted it will take the same promise/failure transition as before its restart --- based on the value of $i.successful\_atomic$.
% \end{commentary}

\paragraph{Initiate memory store operation footprints of store instruction}\label{omm:announce_store_footprint}
An instruction instance $i$ with next pseudocode state {\sc Store\_ea}($kind$, $address$, $size$, $next\_state$) can announce its pending memory store operation footprint.
Action:
\begin{enumerate}
\item construct the appropriate memory store operations $msos$ (without store value):
  \begin{itemize}
  \item if $address$ is aligned to $size$ then $msos$ is a single memory store operation of $size$ bytes to $address$;
  \item otherwise $msos$ is a set of $size$ memory store operations, each of one-byte size, to the addresses $address\ldots address+size-1$.
  \end{itemize}
\item set $i.mem\_stores$ to $msos$; and
\item update the state of $i$ to ``{\sc Plain} $next\_state$''.
\end{enumerate}

\begin{commentary}
Note that after taking the transition above the memory store operations do not yet have their values.
The importance of splitting this transition from the transition below is that it allows other program-order-later store instructions to observe the memory footprint of this instruction, and if they don't overlap, propagate out of order as early as possible (i.e.~before the data register value becomes available).
\end{commentary}


\paragraph{Instantiate memory store operation values of store instruction}\label{omm:initiate_store}
An instruction instance $i$ with next pseudocode state {\sc Store\_memv}($memory\_value$, $store\_continuation$) can instantiate the values of the memory store operations $i.mem\_stores$.
Action:
\begin{enumerate}
\item split $memory\_value$ between the memory store operations $i.mem\_stores$; and
\item update the state of $i$ to ``{\sc Pending\_mem\_stores} $store\_continuation$''.
\end{enumerate}


\paragraph{Commit store operations}\label{omm:commit_stores}
For an uncommitted store instruction $i$ in state ``{\sc Pending\_mem\_stores} $store\_continuation$'', $i$ can be committed if:
\begin{enumerate}
\item $i$ has fully determined data;
\item all program-order-previous conditional branch and indirect jump instructions are finished;
\item all program-order-previous {\em fence} instructions with {\em .sw} set are finished;
\item all program-order-previous {\em fence.i} instructions are finished; \fixme{remove?}
\item all program-order-previous load-acquire instructions are finished;
\item  if $i$ is a store-release, all program-order-previous memory access instructions are finished;
\item\label{omm:commit_store:prev_addrs} all program-order-previous memory access instructions have a fully determined memory footprint;
\item\label{omm:commit_store:prev_stores} all program-order-previous store instructions, except for {\em sc} that failed, have initiated and so have non-empty $mem\_stores$; and
\item\label{omm:commit_store:prev_loads} all program-order-previous load instructions have initiated and so have non-empty $mem\_loads$.
\end{enumerate}
Action: record $i$ as committed.

\begin{commentary}
Notice that if condition \ref{omm:commit_store:prev_addrs} is satisfied the conditions \ref{omm:commit_store:prev_stores} and \ref{omm:commit_store:prev_loads} are also satisfied, or will be satisfied after taking some eager transitions.
Hence, requiring them does not strengthen the model.
By requiring them, we guarantee that previous memory access instructions have taken enough transitions to make their memory operations visible for the condition check of \nameref{omm:prop_store}, which is the next transition the instruction will take, making that condition simpler.
\end{commentary}


\paragraph{Propagate memory store operation}\label{omm:prop_store}
For an instruction $i$ in state ``{\sc Pending\_mem\_stores} $store\_continuation$'', and an unpropagated memory store operation $mso$ in $i.mem\_stores$, the memory store operation can be propagated if:
\begin{enumerate}
\item all memory memory store operations of program-order-previous store instructions that overlap with $mso$ have already propagated;
\item all memory load operations of program-order-previous load instructions that overlap with $mso$ have already been satisfied, and (the load instructions) are non-restartable (see below);
\item all memory load operations satisfied by forwarding $mso$ are entirely satisfied; and
\item no AMO or successful {\em lr} from a different hart has an outstanding lock on a memory store operation slice that overlaps with $mso$.
\end{enumerate}
In the above, instruction $i'$ might be restarted if it is a non-finished instruction and at least one of the following holds:
\begin{enumerate}
\item there exists a store instruction $s$ and an unpropagated memory store operation $mso$ of $s$ such that applying the action of the \nameref{omm:prop_store} transition to $mso$ will result in the restart of $i'$;
\item there exists a non-finished load instruction $l$ and a memory load operation $mlo$ of $l$ such that applying the action of the \nameref{omm:sat_from_mem} transition to $mlo$ will result in the restart of $i'$ (even if $mlo$ is already satisfied); or
\item there exists a non-finished instruction $i''$, program-order-before $i'$, that might be restarted and $i'$ has a data-flow dependency on $i''$, or $i''$ is a load-acquire.
\end{enumerate}
(An instruction is non-restartable if it does not satisfy the condition above.)
Action:
\begin{enumerate}
\item update the memory with $mso$;
\item record $mso$ as propagated;
\item restart any speculative instructions which have violated coherence as a result of this, i.e., for every non-finished instruction $i'$ program-order-after $i$ and every memory load operation $mlo'$ of $i'$ that was satisfied from $msoss'$, if there exists a memory store operation slice $msos'$ in $msoss'$ that overlaps with $mso$ and is not from $mso$, and $msos'$ is not from a program-order-successor of $i$, restart $i'$ and its data-flow dependents; and
\item for every AMO and successful {\em lr} that has read from $mso$ (by forwarding), add the slices of $mso$ this load reads from to the set of memory store operation slices the memory load operation of the load is mapped to in the exclusives map.
\end{enumerate}

\paragraph{Complete store operations}\label{omm:complete_stores}
A store instruction $i$ in state ``{\sc Pending\_mem\_stores} $store\_continuation$'', for which all the memory store operations in $i.mem\_stores$ have been propagated, can be completed (not to be confused with finished).
Action: update the state of $i$ to ``{\sc Plain} $store\_continuation(true)$''.


\paragraph{Commit fence}\label{omm:commit_barrier}
A fence instruction $i$ in state ``{\sc Plain} $next\_state$'' where $next\_state$ is {\sc Fence}($kind$, $next\_state'$) can be committed if:
\begin{enumerate}
\item all program-order-previous conditional branch and indirect jump instructions are finished;
\fixme{this looks stronger than intended, but actually it has no observable effect for most fences; the exception is ``fence w,r'', e.g., MP+fence.w.w+ctrlfence.w.r is allowed by Daniel's model but forbidden as a consequence of this condition. Fixing this means changing the invariant that finished instructions are never discarded, to finished load/store instructions are never discarded. Is RISC-V going to include ``fence w,r''?}
\item if $i$ has {\em .pr} set, all program-order-previous load instructions are finished;
\item if $i$ has {\em .pw} set, all program-order-previous store instructions are finished; and
\item if $i$ is a {\em fence.i} instruction, all program-order-previous memory access instructions have fully determined memory footprints. \fixme{remove?}
\end{enumerate}
Action: update the state of $i$ to ``{\sc Plain} $next\_state'$''.


\paragraph{Register read}\label{omm:reg_read}
An instruction instance $i$ with next pseudocode state {\sc Read\_reg}($reg\_name$, $read\_cont$) can do a register read of $reg\_name$ if every instruction instance that it needs to read from has already performed the expected $reg\_name$ register write.

Let $read\_sources$ include, for each bit of $reg\_name$, the write to
that bit by the most recent (in program order) instruction instance that can write to that bit, if any. If there is no such instruction, the source is the initial register value from $initial\_register\_state$.
Let  $register\_value$ be the value assembled from $read\_sources$.

Action:
\begin{enumerate}
\item add $reg\_name$ to $i.reg\_reads$ with $read\_sources$ and $register\_value$; and
\item update the state of $i$ to ``{\sc Plain} $read\_cont(register\_value)$''.
\end{enumerate}


\paragraph{Register write}\label{omm:reg_write}
An instruction instance $i$ with next pseudocode state {\sc Write\_reg}($reg\_name$, $register\_value$, $next\_state'$) can do a $reg\_name$ register write.
Action:
\begin{enumerate}
\item add $reg\_name$ to $i.reg\_writes$ with $deps$ and $register\_value$; and
\item update the state of $i$ to ``{\sc Plain} $next\_state'$''.
\end{enumerate}
where $deps$ is a pair of the set of all $read\_sources$ from
$i.reg\_reads$, and a flag that is true iff $i$ is a load instruction
that has already been entirely satisfied.\fixme{change wrt success
  register of sc?}


\paragraph{Pseudocode internal step}\label{omm:sail_interp}
An instruction instance $i$ with next pseudocode state {\sc Internal}($next\_state'$) can do that pseudocode-internal step.
Action: Update the state of $i$ to ``{\sc Plain} $next\_state'$''.


\paragraph{Finish instruction}\label{omm:finish}
A non-finished instruction $i$ with next pseudocode state {\sc Done} can be finished if:
\begin{enumerate}
\item if $i$ is a load instruction:
  \begin{enumerate}
  \item all program-order-previous load-acquire instructions are finished; and
  \item it is guaranteed that the values read by the memory load operations of $i$ will not cause coherence violations, i.e., for any program-order-previous instruction instance $i'$, let $cfp$ be the combined footprint of propagated memory store operations from store instructions program-order-between $i$ and $i'$, and fixed memory store operations that were forwarded to $i$ from store instructions program-order-between $i$ and $i'$ including $i'$, and let $cfp'$ be the complement of $cfp$ in the memory footprint of $i$.
  If $cfp'$ is not empty:
    \begin{enumerate}
    \item $i'$ has a fully determined memory footprint;
    \item $i'$ has no unpropagated memory store operations that overlap with $cfp'$; and
    \item if $i'$ is a load with a memory footprint that overlaps with $cfp'$, then all the memory load operations of $i'$ that overlap with $cfp'$ are satisfied and $i'$ is non-restartable (see the \nameref{omm:prop_store} transition for how to determined if an instruction is non-restartable).
    \end{enumerate}
  Here, a memory store operation is called fixed if the store instruction has fully determined data.
  \end{enumerate}
\item $i$ has a fully determined data; and
\item all program-order-previous conditional branch and indirect jump instructions are finished.
\end{enumerate}
Action:
\begin{enumerate}
\item if $i$ is a conditional branch or indirect jump instruction, discard any untaken paths of execution, i.e., remove any (non-finished) instructions that are not reachable by the branch/jump taken in $instruction\_tree$; and
\item record the instruction as finished, i.e., set $finished$ to $true$.
\end{enumerate}


\subsection{Limitations}\label{sec:omm:limitations}
\begin{itemize}
\item The model covers user-level RV64I and the ``A'' standard extension.\fixme{also RV32I?}
\item The model does not cover TLB-related effects.
\item The model does not cover exceptions, traps and interrupts.
The {\tt rmem} tool catches and reports instruction decode errors and memory accesses to memory locations that were not declared in the initial state.
\item The model assumes the instruction memory is fixed.
In particular, the \nameref{omm:fetch} transition does not generate memory load operations, and the shared memory is not involved in the transition.
Instead, the model depends on an external oracle that provides an opcode when given a memory location.
\item \fixme{deadlocks?}
\end{itemize}


